\chapter{Medium of Finite Conductivity}\label{lec:lec25}
\subparagraph{According to Maxwell's equation:}
\begin{equation}
\nabla\cdot D = \rho_{v}\footnotemark
\end{equation}
\footnotetext{
Gauss's Law for the electric field describes the static electric field generated by a distribution of electric charges
}
\begin{equation}	 
\nabla\cdot B = 0\footnotemark
\end{equation}
\footnotetext{
Gauss's Law for magnetism states that the magnetic field B has divergence equal to zero
}
\begin{equation}
\nabla X E = \frac{\delta B}{\delta t}\footnotemark
\end{equation}
\footnotetext{
Faraday's Law states that when the magnetic flux linking a circuit changes, an EMF is induced in the circuit proportional to the rate of change of the flux linkage
}
\begin{equation}
\nabla XH =\frac{\delta \bar{D}}{\delta t} + J\footnotemark
\end{equation}
\footnotetext{
Ampere - Maxwell law relates electric currents and magnetic flux
}
\section{Maxwell's equation in relation to Ampere's Law}
The conductivity of a medium appears in Maxwell's equation and it corresponds to the ampere's law.

We have two cases to deal with:

\textbf{Case 1:} When the conductivity of the medium = 0 

The current density is 0, therefore:

\begin{equation}
\nabla X \bar{H} = \frac{\delta \bar {D}}{\delta t}
\end{equation}

\textbf{Case 2:} We take into consideration, the conduction current density which is related to the finite conductivity of the medium.

\begin{center}
$\nabla X \bar{H} = \frac{\delta \bar{D}}{\delta t} + J$
\end{center}

We know from Ohm's law:
\begin{equation}
J = \sigma\bar{E}\footnotemark
\end{equation}
\footnotetext{
In physics, the term Ohm's law is also used to refer to various generalizations of the law originally formulated by Ohm. Where J is the current density at a given location in a resistive material, E is the electric field at that location, and $\sigma$ is a material-dependent parameter called the conductivity.
}
The simplest case is an Isotropic medium (homogenous) which makes the conductivity uniform in the space and not dependent on the direction.

When an electric field is impressed in a medium of finite conductivity, then there are two types of current charges that will flow in the medium; One corresponds to the current density($\sigma$), and the other corresponds to the displacement current density.

So in general, in any medium, there are two types of current that flow:
\textbf{Conduction current and the Displacement current.}

Let us investigate the equations in both Case 1 and Case 2 above:

Recall:
\begin{equation}
\bar{D} = \epsilon_{o}\epsilon_{r}\bar{E}\footnotemark=\epsilon\bar{E}
\end{equation}
\footnotetext{
Electric flux density is a measure of the strength of an electric field generated by a free electric charge, corresponding to the number of electric lines of force passing through a given area.
}
\begin{equation}
\frac{\delta}{\delta t} = j\omega\footnote{Only systems with sinusoidal excitation would this replacement be possible} 
\end{equation}

\textbf{For a finite conductivity in a medium:}

According to the equation above:

\begin{center}
$\nabla X \bar{H} = \frac{\delta \bar{D}}{\delta t} + J$
\end{center}

Substituting for $\bar{D}$ in the equation above, we have:

\begin{center}
$\nabla X \bar{H} = \frac{\delta\epsilon\bar{E}}{\delta t} + J$
\end{center}

Substituting $\epsilon$ = $\epsilon_{o}\epsilon_{r}$

\begin{equation}
\nabla X \bar{H} = \frac{\delta}{\delta t}[\epsilon_{o}\epsilon_{r}\bar{E}] + J
\label{medium with finite conductivity}
\end{equation}

Substituting $\frac{\delta}{\delta t} = j\omega$ in equation\ref{medium with finite conductivity}

\begin{equation}
\nabla X \bar{H} = j\omega\epsilon_{o}\epsilon_{r}\bar{E} + J
\label{medium with finite conductivity2}
\end{equation}

According to Ohm's law, $J = \sigma\bar{E}$. We substitute this in equation\ref{medium with finite conductivity2} to get:

\begin{center}
$\nabla X \bar{H} = j\omega\epsilon_{o}\epsilon_{r}\bar{E} + \sigma\bar{E}$ 
\end{center}

\begin{center}
$\nabla X \bar{H} = j\omega\epsilon_{o}\{\epsilon_{r} + \frac{\sigma}{j\omega\epsilon_{o}}\}\bar{E}$ 
\end{center}

\begin{equation}
\nabla X \bar{H} = j\omega\epsilon_{o}\Bigg\{\epsilon_{r} -j \frac{\sigma}{\omega\epsilon_{o}}\Bigg\}\bar{E} 
\end{equation}

\textbf{For No conductivity:}

According to the equation above:

\begin{center}
$\nabla X \bar{H} = \frac{\delta \bar {D}}{\delta t}$
\end{center}

Substituting $\frac{\delta}{\delta t} = j\omega$ in the equation above

\begin{center}
$\nabla X \bar{H} = j \omega(\bar{D})$
\end{center}

Substituting for $\bar{D} = \epsilon_{o}\epsilon_{r}\bar{E}$ in the equation above, we have:

\begin{equation}
\nabla X \bar{H} = j \omega(\epsilon_{o}\epsilon_{r}\bar{E})
\end{equation}



Thus,

For a medium with no conductivity:
\begin{center}
$\nabla X \bar{H} = j \omega(\epsilon_{o}\epsilon_{r}\bar{E})$
\end{center}

For a medium of finite conductivity:
\begin{center}
$\nabla X \bar{H} = j\omega\epsilon_{o}\bigg\{\epsilon_{r} -j \frac{\sigma}{\omega\epsilon_{o}}\bigg\}\bar{E}$ 
\end{center}

\section{Complex Dielectric Constant}
Now, when we compare the two equations for no conductivity and for a finite conductivity in a medium, and we treat the medium like a dielectric, we realize that:

\begin{center}
$j \omega(\epsilon_{o}\epsilon_{r}\bar{E}) \equiv j\omega\epsilon_{o}\bigg\{\epsilon_{r} -j \frac{\sigma}{\omega\epsilon_{o}}\bigg\}\bar{E}$ 
\end{center}

$\epsilon_{r} -j \frac{\sigma}{\omega\epsilon_{o}}$ becomes the Relative permittivity(dielectric constant) of the medium.

Because of the finite conductivity, the dielectric constant has become a complex quantity.

\textit{Hence, the conductivity of a medium can be accounted for by effectively introducing the concept of the complex dielectric constant.}

\begin{equation}
\textbf{Complex dielectric constant} (\epsilon_{rc}) = \epsilon_{r} -j \frac{\sigma}{\omega\epsilon_{o}}
\end{equation}

\textit{NOTE: When we have a finite conductivity in a medium, the dielectric constant becomes a complex quantity. The dielectric constant has become a function of frequency.}

Earlier, the medium properties were not dependent on frequency but if we introduce the concept of complex dielectric constant, then the medium property which is the relative permittivity or dielectric constant(which has become a complex quantity) now depends on frequency.

Now, the question that comes to mind is, should we call this medium a conductor or a dielectric? The answer lies in the two terms in the equation for finite conductivity.
\begin{align*}
\nabla \cross \bar{H} = j\omega\epsilon\bar{E} + \sigma\cdot\bar{E} 
\end{align*}
When an electric field is imposed firmly in the medium either the conduction current or displacement current dominates.

\begin{enumerate}[(i)]
\item If the displacement current dominates we say that this medium is a dielectric.

If the conduction current dominates we say this medium is a conductor.

Mathematically;

If $\sigma \gg \omega\epsilon$ then the medium is a Conductor.

If $\sigma \ll \omega\epsilon$ Then the medium is a Dielectric.

\item Since the term $\omega\epsilon$ is dependent on the frequency range and it is dependent on a given permittivity or conductivity, it might behave as a dielectric or conductor, depending on what frequency range we are operating on. 

If the frequency is very low, $\sigma$ would be much larger compared to $\omega\epsilon$.

If the frequency is very high, $\sigma$ would be much lesser compared to $\omega\epsilon$.

\textbf{Therefore, the lower we go on the frequency spectrum, the more conductive the medium becomes and the higher we go on the frequency spectrum, the more dielectric the medium becomes.}

\textit{Hence, the behaviour of the medium depends on the frequency of the operation.}

\item When $\sigma = \omega\epsilon = \omega\epsilon_{o}\epsilon_{r}$

\begin{equation} \omega_{T} = \dfrac{\sigma}{\epsilon_{o}\epsilon_{r}}
\end{equation}
$\omega_{T}$ is called the transitive frequency.

At this point, the conduction current density and the displacement current density become equal.

If $\omega > \omega_{T}$ then it is a dielectric medium.

If $\omega < \omega_{T}$ then it is a conductor.
\end{enumerate}


\begin{exmp}
What are the frequency ranges in which copper can act like a dielectric or conductor?

Copper: $\sigma = 5.6 X 10^{7}S/m$, 
$\epsilon_{o} = 8.854 X 10^{-12}Fm^{-1}$ $\epsilon_{r} = 1$

Recall,
\begin{equation}
\omega_{T} = 2\pi f_{T}
\end{equation}
And $ \omega_{T} = \dfrac{\sigma}{\epsilon_{o}\epsilon_{r}}$, so,
\begin{equation}
2\pi f_{T} = \dfrac{\sigma}{\epsilon_{o}\epsilon_{r}}
\end{equation}

When we substitute the values into the formula we have:

$f_{T} = 10^{18}Hz$

This implies that copper will behave as a Conductor at this frequency and below but if we go higher than this in the frequency spectrum, then copper begins to behave like a dielectric medium.
\end{exmp}

\begin{exmp}
The conductivity of sea water, $\sigma = 10^{-3}S/m$ , $\epsilon_{o} = 8.854 X 10^{-12}Fm^{-1}$ , dielectric constant $\epsilon_{r} = 80$.
When we substitute these values into the formula we have, $f_T=225KHz$

Seawater is like a dielectric. However, at a frequency less than 225KHz, the sea water is more like a conductor.
\end{exmp}

\section{Propagation of Electromagnetic Waves}
We'll take two extreme cases for the propagation of electromagnetic waves for a medium of finite conductivity which are:
\begin{enumerate}[(i)]
\item Low Conductivity (good dielectric)
\item High Conductivity (good conductor)
\end{enumerate}
Maxwell's equation for the two cases above is solved for a dielectric medium by replacing the dielectric constant of this medium with the complex dielectric constant for the conducting medium. 

\subsection{Wave equation}
\begin{equation}
\nabla^{2}
\Bigg\{\bar{E},\bar{H}\Bigg\} = -\omega^{2}\mu\epsilon_{o}\epsilon_{rc}\Bigg\{\bar{E},\bar{H}\Bigg\}\footnotemark  
\end{equation}
\footnotetext{
The wave equation is an important second-order linear partial differential equation for the description of waves - as they occur in classical physics.
}
This means that,
\begin{center}
$\nabla^{2}\bar{E} = -\omega^{2}\mu\epsilon_{o}\epsilon_{rc}\bar{E}$ 
\end{center}
\begin{center}
$\nabla^{2}\bar{H} = -\omega^{2}\mu\epsilon_{o}\epsilon_{rc}\bar{H}$ 
\end{center}
\begin{center}
$\gamma^{2} = -\omega^{2}\mu\epsilon_{o}\epsilon_{rc}$
\end{center}
\begin{center}
$\gamma$ = Propagation constant
\end{center}
Therefore,
\begin{equation}
\textbf{Propagation constant}= \sqrt{-\omega^{2}\mu\epsilon_{o}\epsilon_{rc}\footnotemark
}
\label{propagation constant}
\end{equation}
\footnotetext{
The propagation constant of a sinusoidal electromagnetic wave is a measure of the change undergone by the amplitude and phase of the wave as it propagates in a given direction.
}
Let us assume the medium is not magnetic, ie $\mu_{r} = 1$, then we substitute $\epsilon_{rc} = \epsilon_{r} -j \frac{\sigma}{\omega\epsilon_{o}}$ in equation\ref{propagation constant} and then we have a new equation:
\begin{equation}
\gamma = j\omega\sqrt{\mu\epsilon_{o}}\Bigg\{\epsilon_{r} - j \dfrac{\sigma}{\omega\epsilon_{o}}\Bigg\}^{\frac{1}{2}}
\end{equation}
This is a complex quantity and no longer a purely magnetic quantity. First, we note that $\sigma$ the conductivity of the material is introduced. If the medium was a pure dielectric with no conductivity, $\sigma$ = 0. From $\gamma = \alpha + j\beta$, $\alpha$ tells you the change in amplitude of the wave as the wave travels. This is called \textbf{Attenuation constant.} So as soon as we have conductivity in the medium, we have an attenuation constant. So when the wave propagates in a medium with finite conductivity, its amplitude reduces. Physically when there was no conductivity in the medium, the wave was propagating, we had an electric and magnetic field. Now we have a finite conductivity in the medium, this leads to a finite conduction current. With finite conductivity, we have a finite resistivity as the inverse of conductivity is resistivity. Once you have conductivity, the conduction current flows, and since the medium has finite resistivity, then we have an ohmic loss in the medium or $I^2 R$ loss in the medium. As a result, when the wave propagates, the power or energy which the wave is carrying gets converted into heat. It heats up the medium as a result. This is the reason why as the wave propagates, its amplitude reduces, because the power carried is reduced as it travels along the medium. Physically, it makes sense that when we have a finite conductivity, we must have an attenuation of the wave. 

\textbf{Attenuation Constant:}
\begin{center}
$\alpha = Re(\gamma) = Re\Bigg\{j\omega\sqrt{\mu\epsilon_{o}}\bigg\{\epsilon_{r} - j\dfrac{\sigma}{\omega\epsilon_{o}}\bigg\}^{\frac{1}{2}}\Bigg\}$
\end{center}

\begin{center}
$\beta = I_{m}(\gamma) = I_{m}\Bigg\{j\omega\sqrt{\mu\epsilon_{o}}\bigg\{\epsilon_{r} - j\dfrac{\sigma}{\omega\epsilon_{o}}\bigg\}^{\frac{1}{2}}\Bigg\}$
\end{center}

The result of the algebraic manipulation is:

\begin{equation}
\alpha = \omega\sqrt{\dfrac{\mu_{o}\epsilon_{o}\epsilon_{r}}{2}}\Bigg\{{\sqrt{1 + \dfrac{\sigma^{2}}{\omega^{2}\epsilon_{o}^{2}\epsilon_{r}^{2}}}} - 1\Bigg\}^{\frac{1}{2}}\footnotemark
\end{equation}
\footnotetext{
The real part of the propagation constant is the attenuation constant and is denoted by $\alpha$(alpha). It causes signal amplitude to decrease along a transmission line. The natural units of the attenuation constant are Nepers/meter
}
\begin{equation}
\beta = \omega\sqrt{\dfrac{\mu_{o}\epsilon_{o}\epsilon_{r}}{2}}\Bigg\{{\sqrt{1 + \dfrac{\sigma^{2}}{\omega^{2}\epsilon_{o}^{2}\epsilon_{r}^{2}}}} + 1\Bigg\}^{\frac{1}{2}}\footnotemark
\end{equation}
\footnotetext{
The imaginary part of the propagation constant is the phase constant and is denoted by $\beta$(beta).
}
We will show how we got this result above and some procedure would be valid to get $\alpha$ and $\beta$

\begin{equation}
\gamma^{2} = -\omega^{2}\mu\epsilon_{o}\bigg\{\epsilon_{r} - j \dfrac{\sigma}{\omega\epsilon_{o}}\bigg\}
\end{equation}

\begin{center}
$\gamma = \alpha + j\beta$
\end{center}

\begin{center}
$(\alpha + j\beta)^{2} = -\omega^{2}\mu\epsilon_{o}\bigg\{\epsilon_{r} - j \dfrac{\sigma}{\omega\epsilon_{o}}\bigg\}$
\end{center}

\begin{center}
$(\alpha^{2} - \beta^{2} + j2\alpha\beta) \equiv -\omega^{2}\mu\epsilon_{o}\bigg\{\epsilon_{r} - j \dfrac{\sigma}{\omega\epsilon_{o}}\bigg\}$
\end{center}

Open the bracket and compare the real terms and the imaginary terms and then we have:

\begin{center}
$\alpha^{2} - \beta^{2} = -\omega^{2}\mu\epsilon_{o}\epsilon_{r}$
\end{center}

\begin{center}
2$\alpha\beta = \mu\omega\sigma\Rightarrow \alpha = \dfrac{\mu\omega\sigma}{2\beta}$ 
\end{center}

Put $\alpha$ in the equation above:

\begin{center}
$\bigg\{\dfrac{\mu\omega\sigma}{2\beta}\bigg\}^{2} - \beta^{2} = -\omega^{2}\mu\epsilon_{o}\epsilon_{r}$
\end{center}

\begin{center}
$\dfrac{\mu^{2}\omega^{2}\sigma^{2}}{4\beta^{2}} - \beta^{2} = -\omega^{2}\mu\epsilon_{o}\epsilon_{r}$
\end{center}

\begin{center}
$\mu^{2}\omega^{2}\sigma^{2} - 4\beta^{4} = -4\beta^{2}\omega^{2}\mu\epsilon_{o}\epsilon_{r}$
\end{center}

Multiply through by the negative sign:

\begin{center}
$4\beta^{4} -4\beta^{2}\omega^{2}\mu\epsilon_{o}\epsilon_{r} - \mu^{2}\omega^{2}\sigma^{2} = 0$
\end{center}

Then, let $\beta^{2}$ = x
\begin{center}
$4x^{2} -4x\omega^{2}\mu\epsilon_{o}\epsilon_{r} - \mu^{2}\omega^{2}\sigma^{2} = 0$
\end{center}
The equation above follows the general expression of a quadratic equation of the form $ax^2 + bx + c = 0$
\\
From the quadratic formula:

\begin{center}
$x = \dfrac{-b\pm \sqrt{b^{2} - 4ac}}{2a}$
\end{center}

Where a = 4, b = $-4\omega^{2}\mu\epsilon_{o}\epsilon_{r}$ , c = $\mu^{2}\omega^{2}\sigma^{2}$

\begin{center}
$x = \dfrac{4\omega^{2}\mu\epsilon_{o}\epsilon_{r} + \sqrt{16\omega^{4}\mu^{2}\epsilon_{o}^{2}\epsilon_{r}^{2} + 16\mu^{2}\omega^{2}\sigma^{2}}}{8}$
\end{center}

\begin{center}
$x = \dfrac{\omega^{2}\mu\epsilon_{o}\epsilon_{r} + \sqrt{\omega^{4}\mu^{2}\epsilon_{o}^{2}\epsilon_{r}^{2} + \mu^{2}\omega^{2}\sigma^{2}}}{2}$
\end{center}

Now,
\begin{center}
$\beta^{2} = x = \dfrac{\omega^{2}\mu\epsilon_{o}\epsilon_{r}}{2}\Bigg\{1 + \bigg\{1 + \dfrac{\sigma^{2}}{\omega^{2}\epsilon_{o}^{2}\epsilon_{r}^{2}}\bigg\}^{\frac{1}{2}}\Bigg\}$
\end{center}

Then,

\begin{center}
$\beta =\Bigg\{ \dfrac{\omega^{2}\mu\epsilon_{o}\epsilon_{r}}{2}\bigg\{1 + (1 + \dfrac{\sigma^{2}}{\omega^{2}\epsilon_{o}^{2}\epsilon_{r}^{2}})^{\frac{1}{2}}\bigg\}\Bigg\}^{\frac{1}{2}}$
\end{center}

With relative permeability $\mu_{r}$ = 1, $\mu = \mu_{o}$

\begin{center}
$\beta = \omega\sqrt{\dfrac{\mu_{o}\epsilon_{o}\epsilon_{r}}{2}}\Bigg\{{\sqrt{1 + \dfrac{\sigma^{2}}{\omega^{2}\epsilon_{o}^{2}\epsilon_{r}^{2}}}} + 1\Bigg\}^{\frac{1}{2}}$
\end{center}

$\alpha$ can be obtained in a similar way from $\alpha^{2} - \beta^{2} = -\omega^{2}\mu\epsilon_{o}\epsilon_{r}$

\begin{center}
$\alpha = \omega\sqrt{\dfrac{\mu_{o}\epsilon_{o}\epsilon_{r}}{2}}\Bigg\{{\sqrt{1 + \dfrac{\sigma^{2}}{\omega^{2}\epsilon_{o}^{2}\epsilon_{r}^{2}}}} - 1\Bigg\}^{\frac{1}{2}}$
\end{center}

\textit{NOTE: The attenuation constant is a function of frequency $\omega$ and the conductivity of the medium. It is also true for the phase constant. With pure dielectric, $\beta$ was only proportional to frequency $\omega$. So in the presence of finite conductivity, the attenuation constant,$\alpha$ and phase constant $\beta$ of the waves have become functions of frequency and conductivity.}

Hence, we take the two extreme cases with this investigation made.
If we take the conductivity which is very small compared to $\omega\epsilon$: A dielectric medium (low loss medium) has a conductivity and if the conductivity is more, the conduction current increases, therefore the loss will increase. In the normal electrical circuits we are used to, when the conductivity increases, there will be less loss in the circuit. More conductivity means less resistivity and the smaller the loss in the circuit. However, what we are seeing here is quite the opposite. We find that when the conductivity is higher, we have a higher attenuation, and thus more loss of energy in the medium.

\section{Loss Tangent}

When we are dealing with electrical circuits, we are dealing with conduction currents. We were dealing with components that were more like conducting load of components and in that situation when the conductivity is large or infinite (for a conductor) there is no loss, the resistivity is zero and even if the current flows, there is no loss of power.
However, when we come to the case of a dielectric, then an ideal dielectric without any conductivity does not have any loss. So if we have a medium which is like a dielectric(low loss), it must have zero conductivity. In both situations, there is no loss. Then if we take a dielectric medium, the higher the conductivity, the higher the loss. If we take an ideal conductor, the lower the conductivity, the more the loss. So essentially, what we see here is that, since we are now dealing with a dielectric medium, the increase in the conductivity of the medium essentially gives us the Ohmic loss and because of that we have attenuation and a power loss in the medium.
As a measure of how much power is lost in the medium, we define a parameter called the LOSS TANGENT which is the ratio of conduction current to displacement current.

\begin{equation}
\textbf{Loss tangent}(\tan\delta) = \dfrac{\sigma}{\omega\epsilon_{o}\epsilon_{r}}
\end{equation}

So, whenever we talk about a dielectric, how good the dielectric is, is measured by the loss tangent. Since we are talking about a good dielectric, we assume that this medium is predominantly dielectric and the conduction current is much smaller than the displacement current. So generally, for a good dielectric material, the loss tangent is extremely small. 

\begin{center}
$(\tan\delta) = \dfrac{\sigma}{\omega\epsilon_{o}\epsilon_{r}} \ll 1$ for a good dielectric
\end{center}

So, if we take a material which we want to use for dielectric, then $\tan\delta$ should be very small. Ideally we want $\tan\delta$ = 0, but in practice it ranges from $10^{-4}$ to $10^{-3}$. The smaller the value, the better the dielectric, and the less the ohmic losses as the wave propagates in this medium.\\ 
Whenever we have a dielectric material, the loss tangent is mentioned instead of the frequency and $\tan\delta$ is a frequency-dependent quantity. So we must know the loss tangent at the frequency of operation. If we know these values at a certain frequency, we can always compare the loss tangent to the appropriate frequency. This is a useful parameter for characterizing dielectric materials and determining whether it is a good dielectric or not.\\ 
Let us now take the two extreme cases of wave propagation that is, when the medium is a very good dielectric and when it is a very good conductor.

For a good dielectric (low loss dielectric): $\omega\epsilon_{o}\epsilon_{r} \gg \sigma$

\begin{center}
$\alpha = \omega\sqrt{\dfrac{\mu_{o}\epsilon_{o}\epsilon_{r}}{2}}\Bigg\{{\sqrt{1 + \dfrac{\sigma^{2}}{\omega^{2}\epsilon_{o}^{2}\epsilon_{r}^{2}}}} - 1\Bigg\}^{\frac{1}{2}}$
\end{center}

\begin{center}
$\alpha = \omega\sqrt{\dfrac{\mu_{o}\epsilon_{o}\epsilon_{r}}{2}}\Bigg\{\bigg\{1 + \dfrac{1}{2} \dfrac{\sigma^{2}}{\omega^{2}\epsilon_{o}^{2}\epsilon_{r}^{2}} + \dots\bigg\} - 1\Bigg\}^{\frac{1}{2}}$
\end{center}

\begin{center}
$\alpha = \omega\sqrt{\dfrac{\mu_{o}\epsilon_{o}\epsilon_{r}}{2}}\Bigg\{\dfrac{\sigma}{\sqrt{2}\omega\epsilon_{o}\epsilon_{r}}\Bigg\} =\dfrac{\sigma}{2}\sqrt{\dfrac{\mu_{o}}{\epsilon_{o}\epsilon_{r}}}$
\end{center}

Now for $\beta$, Recall:
\begin{center}
$\beta = \omega\sqrt{\dfrac{\mu_{o}\epsilon_{o}\epsilon_{r}}{2}}\Bigg\{{\sqrt{1 + \dfrac{\sigma^{2}}{\omega^{2}\epsilon_{o}^{2}\epsilon_{r}^{2}}}} + 1\Bigg\}^{\frac{1}{2}}$
\end{center}

Applying binomial expansion to the term $\sqrt{1 + \frac{\sigma^2}{\omega^2\epsilon_{0}^2\epsilon_{r}^2}}$\\

$\sqrt{1 + \frac{\sigma^2}{\omega^2\epsilon_{0}^2\epsilon_{r}^2}} = \bigg\{1 +\frac{\sigma^2}{\omega^2\epsilon_{0}^2\epsilon_{r}^2}\bigg\}^{\frac{1}{2}} = 1 + \dfrac{\sigma^{2}}{2\omega^{2}\epsilon_{o}^{2}\epsilon_{r}^{2}} + \dots$\\

\begin{center}
$\beta = \omega\sqrt{\dfrac{\mu_{o}\epsilon_{o}\epsilon_{r}}{2}}\Bigg\{1 + \dfrac{\sigma^{2}}{2\omega^{2}\epsilon_{o}^{2}\epsilon_{r}^{2}} + \dots + 1\Bigg\}^{\frac{1}{2}}$
\end{center}

\begin{center}
$\beta = \omega\sqrt{\dfrac{\mu_{o}\epsilon_{o}\epsilon_{r}}{2}}\Bigg\{2 + \dfrac{\sigma^{2}}{2\omega^{2}\epsilon_{o}^{2}\epsilon_{r}^{2}}\Bigg\}^{\frac{1}{2}}$
\end{center}

\begin{center}
$\beta = \omega\sqrt{2}\cdot\sqrt{\dfrac{\mu_{o}\epsilon_{o}\epsilon_{r}}{2}}\Bigg\{1 + \dfrac{\sigma^{2}}{4\omega^{2}\epsilon_{o}^{2}\epsilon_{r}^{2}}\Bigg\}^{\frac{1}{2}}$
\end{center}

\begin{center}
$\beta = \omega\sqrt{\mu_{o}\epsilon_{o}\epsilon_{r}}\Bigg\{1 + \dfrac{\sigma^{2}}{8\omega^{2}\epsilon_{o}^{2}\epsilon_{r}^{2}} + \dots\Bigg\}$
\end{center}


\begin{center}
$\beta = \omega\sqrt{\mu_{o}\epsilon_{o}\epsilon_{r}}\Bigg\{1 + \dfrac{1}{8} \dfrac{\sigma^{2}}{\omega^{2}\epsilon_{o}^{2}\epsilon_{r}^{2}}\Bigg\} \cong \omega\sqrt{\mu_{o}\epsilon_{o}\epsilon_{r}}$
\end{center}

So for $\omega\epsilon_{o}\epsilon_{r} \gg \sigma$, $\alpha$ is very close to zero because $\dfrac{\sigma}{2}\sqrt{\dfrac{\mu_{o}}{\epsilon_{o}\epsilon_{r}}}$ is very small. But however, since $\alpha$ is now accounting for the losses in the medium, no matter how small, we want to know its value. The phase constant here is the same as what we have gotten for the lossless medium or dielectric without finite conductivity. So this case is similar to the case of low-loss transmission lines when compared.
In the low loss Transmission line case, we have said that the phase constant is the same as that of a lossless transmission line, and the attenuation constant is very small. So lines can be treated like lossless lines. Only when the losses are to be calculated, then we take into account the attenuation constant $\alpha$. We can do the same thing for this medium. We can say that for a medium which is low loss ($\omega\epsilon_{o}\epsilon_{r} \gg \sigma$), it can be treated like a lossless medium without conductivity for all practical purposes. Only when we are to find out the amplitude of the wave we then make use of attenuation constant $\alpha$ and find the resultant amplitude over some distance. With $\dfrac{\sigma}{2}\sqrt{\dfrac{\mu_{o}}{\epsilon_{o}\epsilon_{r}}}$, for small values of $\sigma$, $\alpha$ almost increases linearly with $\sigma$. The wave amplitude decays exponentially with $\alpha$ (by $e^{-\alpha x}$). If $\alpha$ varies linearly with $\sigma$, the wave decays rapidly if the value of $\sigma$ changes. So any small change in $\sigma$ will affect the amplitude substantially over some distance. 
We can also define the intrinsic impedance $\eta$ for this medium as:
$\eta= \sqrt{\dfrac{j\omega\mu}{j\omega\epsilon}}$ , which now becomes $\sqrt{\dfrac{j\omega\mu}{\sigma + j\omega\epsilon}}$, and for a good dielectric, $\omega\epsilon_{o}\epsilon_{r} \gg \sigma$ . Now the intrinsic impedance is no more a real quantity when we have finite conductivity in the medium. For a free space, recall that $\eta_{o}$= $\eta$ = 377 or 120$\pi\Omega$ and the medium was appearing like a resistance for a wave propagating in the medium. It felt as if it was going inside a resistance. This is not true when we have finite conductivity. In a medium with finite conductivity, it sees an impedance which is a complex quantity and this is precisely what we saw for the transmission line case also, when $R$ and $G$ were not zero, the characteristic impedance of the Transmission line became complex.
