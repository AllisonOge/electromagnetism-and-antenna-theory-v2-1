\chapter{Investigation of reflection at media interface}
Here we investigate how much energy is reflected and how much is transmitted at the media interface.we had considered lossless medium in d previous lectures for which $ \sigma $=0, but with different permeability and permittivity.\\
We already established that knowing knowing $E_{1}$ it can also be found from $\frac{E}{H} = \eta$ relation.\\
\begin{figure}[h]
	\centering
	\includegraphics[width=1\linewidth]{11}
	\caption{}
\end{figure}
 Again E at arbitrary angle to plane of incidence is resolved to parallel polarized and perpendicular polarized I.e along plane of incidence  and normal to plane of incident respectively.\\
 so write down the electric and magnetic field for the two media and then apply boundary condition at the interface and by some algebra we get $\frac{E_{r}}{E_{i}}$ and $\frac{E_{t}}{E_{i}}$ for reflection coefficient  and transmission  coefficient  respectively.\\ So we find out the $\Gamma$ and $\tau$ for each of the two cases of parallel and perpendicular  polarization for each of the two cases.
\section{Perpendicular polarization} 
From the diagram below the wave is incident at an angle $\theta_{i}$ to the normal. The wave vector lies on the paper or the plane of incidence.\\


For perpendicular  polarization  either the electric field is coming out of the paper in the positive y direction or going into the paper in the negative  y direction.\\
\begin{figure}[h]
	\centering
	\includegraphics[width=1\linewidth]{12}
	\caption{}
\end{figure}
 We can argue that to satisfy the boundary condition, at the interface, the tangential component  of electric field across the boundary should be continuous. If $\bar{E}_{i}$ is y oriented both  $\bar{E}_{r}$ and  $\bar{E}_{t}$ must be y oriented.\\
\begin{figure}[h]
	\centering
	\includegraphics[scale=0.2]{13}
	\caption{}
\end{figure}
 $\bar{E}_{r}$ and  $\bar{E}_{t}$ may not necessarily be in the positive y direction but it is in the y direction . let us just say it is in the positive  y direction so  $\bar{E}_{i}$,  $\bar{E}_{t}$ and  $\bar{E}_{r}$ are all assumed to be coming out of the plane of the paper.we have to write down the corresponding  magnetic field for which we use the pointing Vector argument.we must choose direction of magnetic  field H such that the pointing vector will be in the direction of the wave propagation or wave vector. Since E for all 3 electric field is perpendicular  to the plane of the paper, the H field vector for all 3 must lie in the place of the paper or plane of incident. They can go in direction 1 or 2 shown in the diagram as far as H is on d plane of paper. 1 or 2 will be decided by pointing vector of  $\bar{E} \times \bar{H}$ giving the wave direction since we are dealing with transverse  electromagnetic  waves $\frac{E}{H} = \eta$. So that $\frac{E_{i}}{H_{i}} = \eta_{1}$, $\frac{E_{r}}{H_{r}} = \eta_{1}$, $\frac{E_{t}}{H_{t}} = \eta_{2}$.\\
Again we decompose magnetic  field into component along x and z. E is already perpendicular to the plane of the paper. We bring it out clearly below\\
\begin{figure}[h]
	\centering
	\includegraphics[width=1\linewidth]{14}
	\caption{}
\end{figure}

\begin{center}
	from the diagram(2.4) the following can be evaluated\\
	$ \bar{E_{i}} = E_{1} \varrho^{-j\beta_{1}} (x sin\theta_{i} + z cos\theta_{i}) \hat{y}$\\
	
	
	$ \bar{E_{r}} = E_{r}  \varrho^{-j\beta_{1}} (x sin\theta_{i} - z cos\theta_{i}) \hat{y}$\\
	
	
	$ \bar{E_{t}} = E_{t} \varrho^{-j\beta_{2}} (x sin\theta_{t} + z cos\theta_{t}) \hat{y}$
\end{center}


All electric field has been written out explicitly 

If they have to satisfy the boundary  condition and since all 3 field are tangential  to the boundary, as dielectric boundary, Z=0.the sum of Ei and Er should be equal to $E_{t}$. From the boundary condition, tangential component of E should be continuous  at Z=0 I.e the inter-phase\\ 
\begin{center}
	$E_{i} \varrho^{-j \beta_{i} x sin\theta_{i}} + E_{r} \varrho^{-j \beta_{1} x sin\theta_{1}} = E_{t} \varrho^{-j \beta_{2} x sin\theta_{2}}$
\end{center}
we recall from smells law that $\beta_{i} x sin\theta_{i} = \beta_{1} x sin\theta_{1} = \beta_{2} x sin\theta_{2}$ .
Hence the phase function is the same for all the 3 waves,hence the phase terms is a common terms.\\
This implies that $ E_{i} + E_{r} = E_{t}$.This relationship is true weather  for normal or tangential component,since this phase gradient is same. So the component we take for electric or magnetic field  is immaterial, as this phase function remains same.do whatever boundary condition is satisfy,this phase function  is always  the same so that the boundary  condition can be applied only on the amplitude term. 
\begin{equation}
\centering
 E_{i} + E_{r} = E_{t}
\end{equation}
Since we are talking about dielectric media,there are no surface current,we can also use the continuity of the tangential  component of magnetic field. The tangential component continuity can not be assume if there is possibility of surface current. As we have seen surface current is for conductors. So with a dielectric boundary like  this no surface current, the continuity of tangential component  of magnetic field can be applied. So that 
\begin{center}
	$H_{i} cos\theta_{i} - H_{r} cos\theta_{i} = H_{t} cos\theta_{t}$
\end{center}
\begin{equation}
\frac{E_{i}}{\eta_{1}} cos\theta_{i} - \frac{E_{r}}{\eta_{1}} cos\theta_{i} = \frac{E_{t}}{\eta_{2}} cos\theta_{t}
\end{equation}
We are interested in finding out two component $\Gamma = \frac{E_{r}}{E_{i}}$ and $ \tau = \frac{E_{t}}{\eta_{i}}$, we use this two equation to find out this quantities 
REFLECTION COEFFICIENT FOR PERPENDICULAR POLARIZATION\newline
\\
 $\Gamma_{\bot} = \frac{\eta_{2} cos\theta_{i} - \eta_{1} cos\theta_{t}}{\eta_{2} cos\theta_{i} + \eta_{1} cos\theta_{t}}$\\
\\
 AND TRANSMISSION COEFFICIENT\newline
\\
 \\
  $\tau_{T} = \frac{2 \eta_{2} cos\theta_{i}}{\eta_{2} cos\theta_{i} + \eta_{1} cos\theta_{2}}$\\
dividing equation 2.1 by $H_{r}$ , we have\\
\begin{equation}
\centering
\frac{E_{i}}{H_{r}} + \frac{E_{r}}{H_{r}} = \frac{E_{t}}{H_{r}}
\end{equation}

\begin{center}
	or
$1 + \Gamma = \tau$
\end{center}

\begin{center}
$\frac{1}{\eta_{1}}cos\theta_{i} - \frac{\Gamma}{\eta_{1}}cos\theta_{i} = \frac{\tau}{\eta_{2}}cos\theta_{t}$ but $1 + \Gamma = \tau$\newline
\\
$\frac{1}{\eta_{1}}cos\theta_{i} - \frac{\Gamma}{\eta_{1}}cos\theta_{i} = \frac{1+\Gamma}{\eta_{2}}cos\theta_{t} = \frac{1}{\eta_{2}}cos\theta_{t} + \frac{\Gamma}{\eta_{2}}cos\theta_{t}$\newline
\\
$\frac{1}{\eta_{1}}cos\theta_{i} - \frac{1}{\eta_{2}}cos\theta_{t} = \Gamma( \frac{1}{\eta_{2}}cos\theta_{t} + \frac{1}{\eta_{1}}cos\theta_{i})$\newline
\\
$\eta_{2}cos\theta_{i} - \eta_{1}cos\theta_{t} = \Gamma(\eta_{1}cos\theta_{t} + \eta_{2}cos\theta_{i})$\newline
\\
$\Gamma = \frac{\eta_{2}cos\theta_{i} - \eta_{1}cos\theta_{t}}{\eta_{2}cos\theta_{i} + \eta_{1}cos\theta_{t}}$ ,\newline
\\
 $\tau = 1 + \Gamma = 1 + \frac{\eta_{2}cos\theta_{i} - \eta_{1}cos\theta_{t}}{\eta_{2}cos\theta_{i} + \eta_{1}cos\theta_{t}}$\newline
 \\
 $\tau = \frac{\eta_{2}cos\theta_{i} + \eta_{1}cos\theta_{t} + \eta_{2}cos\theta_{i} - \eta_{1}cos\theta_{t}}{\eta_{2}cos\theta_{i} + \eta_{1}cos\theta_{t}} = \frac{2 \eta_{2}cos\theta_{i}}{\eta_{2}cos\theta_{i} + \eta_{1}cos\theta_{t}}$\newline
 \\
 hence for the perpendicular polarization

\end{center}
\begin{equation}
\centering
\Gamma_{\perp} = \frac{\eta_{2}cos\theta_{i} - \eta_{1}cos\theta_{t}}{\eta_{2}cos\theta_{i} + \eta_{1}cos\theta_{t}}
\end{equation}
\begin{equation}
\centering
\tau_{\perp} = \frac{2 \eta_{2}cos\theta_{i}}{\eta_{2}cos\theta_{i} + \eta_{1}cos\theta_{t}}
\end{equation}

so just after the boundaries,if we find out what are the electric fields,the reflection coefficient gives us what will be the electric field before the boundary  and transmission  coefficient  gives us what  will be the electric field just after  the boundary.\\
Once $E_{r}$ and $E_{t}$ are obtained, then we have the phase function. We can find out the wave at any location at any point in space in medium1 and in medium 2. In medium  1, we have superposition  of the wave $E_{r}$ and $E_{i}$.\\
In medium 2 we have only $E_{t}$. Two things are observed for transmission band and reflection  coefficient\\
(1). the reflection coefficient is always  less than 1 or $|\Gamma_{\bot}| \leq 1$ but $\tau_{\bot}$ could be less than 1 or greater than 1. The pointing vector is proportional  to $|\bar{E}|^{2}$. So if $|\Gamma|< 1$,that means the pointing vector magnitude  for the reflected  wave is always  less  than  1. So the power density of the reflected wave is always  going  to be less than that of the incident wave.  $\tau_{\bot} > 1$ means electric field in medium 2 will be larger than that of the incident electric field. This does not mean that the pointing vector in 2(representing power density) is more than that of that of medium 1.\\
Although electric field is larger in medium 2, the intrinsic impedance $\eta=\sqrt{\frac{\mu}{\epsilon}}$.\\
it is less in medium 2 since $\frac{E}{\eta} =H$. The law of conservation of energy must hold. The power density incident must be equal to the power density transmitted plus power density reflected.\\
Though the electric field in medium 2 will be larger, their poynting vector will be less compared to that for incident wave.\\
When a wave is reflected from the boundary, this may be a phase reversal for the wave or there may be no phase reversal for the wave. The incident  wave and transmitted  wave are always  in phase at $\tau$ Is always  positive, $\Gamma$ can be positive or negative.\\ 
Hence $E_{i}$ and $E_{r}$ can be in phase ($\Gamma_{\perp}$ positive) or out of phase i.e $\Gamma_{\perp}$ is negative. This  means that $E_{i}$ and $E_{t}$ will  always be coming out of the paper, where as for the reflected wave $E_{r}$ and $E_{i}$ can be coming out of the paper(when $\Gamma_{\perp}$ is positive).\\
 $E_{r}$ can be going into the paper(while $E_{i}$ is coming out) for $\Gamma_{\perp}$ negative. This are the important conclusion which we can draw for this wave which is perpendicularly polarized

\section{Parallel polarization}
With H all coming out of the paper E lies on the paper or on the plane of incidence. The phase function is same and there is continuity  of tangential component of magnetic  field.
We get $H_{i} + H_{r} = H_{t}$ or\\
\begin{figure}[h]
	\centering
	\includegraphics[width=1\linewidth]{15}
	\caption{}
\end{figure}
\begin{equation}
\centering
\frac{E_{i}}{\eta_{1}} + \frac{E_{r}}{\eta_{1}} = \frac{E_{t}}{\eta_{2}}
\end{equation}
\begin{equation}
\centering
E_{i} cos\theta_{i} - E_{r} cos\theta_{i} = E_{t} cos\theta_{t}
\end{equation}
divide equation 2.6 by $E_{i}$ we have
\begin{equation}
\centering
\frac{1}{\eta_{1}} + \frac{\Gamma}{\eta_{1}} = \frac{\tau}{\eta_{2}}
\end{equation}
divide equation 2.7 by $E_{i}$ to have
\begin{equation}
cos\theta_{i} - \Gamma cos\theta_{i} = \tau cos\theta_{t}
\end{equation}
\begin{center}
$\tau = \frac{1}{cos\theta_{t}} (cos\theta_{i} + \Gamma cos\theta_{i})$
\end{center}
substituting in equation 2.8 to have
\begin{center}
$\frac{1}{\eta_{1}} + \frac{\Gamma}{\eta_{1}} = \frac{1}{\eta_{2}} \frac{1}{cos\theta_{t}} (cos\theta_{i} + \Gamma cos\theta_{i}) $\newline
\\
$ \frac{1}{\eta_{2}} - \frac{cos\theta_{i}}{\eta_{2} cos\theta_{t}} = - \frac{\Gamma}{\eta_{1}} - \frac{\Gamma cos\theta_{i}}{\eta_{2} cos\theta_{t}}$\newline
\\
$\Gamma (\frac{1}{\eta_{1}} + \frac{cos\theta_{i}}{\eta_{2} cos\theta_{t}}) = \frac{cos\theta_{i}}{\eta_{2} cos\theta_{t}} - \frac{1}{\eta_{1}}$\newline
\\
$\Gamma (\frac{\eta_{2} cos\theta_{t} + \eta_{i} cos\theta_{i}}{\eta_{1} \eta_{2} cos\theta_{t}}) = \frac{\eta_{1} cos\theta_{i} + \eta_{2} cos\theta_{t}}{\eta_{1} \eta_{2} cos\theta_{t}}$\newline
\\
$\Gamma = \frac{\eta_{1} cos\theta_{1} - \eta_{2} cos\theta_{t}}{\eta_{2} cos\theta_{t} + \eta_{1} cos\theta_{1}}$\newline
\\
therefore,
\begin{equation}
\centering
\Gamma_{\parallel} = \frac{\eta_{1} cos\theta_{1} - \eta_{2} cos\theta_{t}}{\eta_{2} cos\theta_{t} + \eta_{1} cos\theta_{1}}
\end{equation}

if $\frac{1}{\eta_{1}} + \frac{\Gamma}{\eta_{1}} = \frac{\tau}{\eta_{2}}$ then\newline
\\
$\tau = \frac{\eta_{2}}{\eta_{1}} (1 + \Gamma)$\newline
\\
$\tau = \frac{\eta_{2}}{\eta_{1}} (1 + \frac{\eta_{1} cos\theta_{1} - \eta_{2} cos\theta_{t}}{\eta_{2} cos\theta_{t} + \eta_{1} cos\theta_{1}})$\newline
\\
$\tau = \frac{\eta_{2}}{\eta_{1}} (\frac{2\eta_{1} cos\theta_{i}}{\eta_{2} cos\theta_{t} + \eta_{1} cos\theta_{i}})$\newline
\\
$\tau = \frac{2 cos\theta_{i} \eta_{2}}{\eta_{2} cos\theta_{t} + \eta_{1} cos\theta_{i}}$\newline
\\
therefore,

\end{center}
\begin{equation}
\centering
\tau_{\parallel} = \frac{2 cos\theta_{i} \eta_{2}}{\eta_{2} cos\theta_{t} + \eta_{1} cos\theta_{i}} 
\end{equation}

The expression we got for reflection and transmission coefficient in perpendicular  and parallel  polarization  are similar except  that $\eta_{1}$ and $\eta_{2}$ are interchanged. All the argument in perpendicular  polarization  regarding  electric  field and pointing vector of power density are valid for the parallel  polarization  also.suppose we have $\Gamma_{\perp}$, $\tau_{T}$ and $\Gamma_{\parallel}$ , $\tau_{\parallel}$ for this two Case we can combine the reflected and transmitted fields,we get the resultant field for reflected or transmitted field for any arbitrary  polarization 
In summary if we have any arbitrary  wave polarization, this arbitrary  polarization  can be resolved into two orthogonal polarization. In this case we take tow linear orthogonal  polarization, parallel  and perpendicular  to the plane  of incidence. Solve the problem separately for This two states of polarization I.e find out $\Gamma_{\parallel}$  $\tau_{\parallel}$ and $\Gamma_{\bot}$ $\tau_{\perp}$ . We can combine the perpendicular And  horizontal polarization to get the resultant electric field or electric polarization. From the expression $\Gamma_{\parallel}$  $\tau_{\parallel}$ and $\Gamma_{\bot}$ $\tau_{\bot}$ we can find out how much field is induced in a second medium when a wave is incidence on a dielectric  boundary. From this weekend can calculate  how much power gets transferred to the secondary medium  and how much power is reflected from the medium. 
Now we consider the special case of perpendicular polarization with  $\theta_{i}$=0. That is direction of propagation if electric  field is normal to plane of incidence
\section{Normal incidence} ($\theta_{i}$ = $\theta_{t}$ = 0)
\begin{figure}[h]
	\centering
	\includegraphics[scale=0.8]{16}
	\caption{}
\end{figure}
at the condition $\theta_{i}$ = $\theta_{t}$ = 0 from equation 2.4 and 2.5 we have\\
\begin{equation}
\centering
\Gamma_{\perp} = \frac{\eta_{2} - \eta_{1}}{\eta_{2} + \eta_{1}}
\end{equation}
\begin{equation}
\tau_{\perp} = \frac{2 \eta_{2}}{\eta_{2} + \eta_{1}}
\end{equation}
we get 
We are not bothered about the variation of the field perpendicular\\  to the transmission line. In this normal incidence case, perpendicular to the direction in which the wave is traveling, there is no field variation. At the interface looking rightward beyond the boundary, we see an infinite medium ahead of intrinsic impedance $\eta_{2}$. looking from the boundary to the right, we see another infinite medium with intrinsic impedance $\eta_{1}$. Hence it is like having two transmission line having two characteristics impedance  $\eta_{2}$ and  $eta_{1}$.when the wave is incident from the left,it is like having a transmission  line with characteristics impedance  $\eta_{1}$ terminated in  $\eta_{2}$\\
\begin{figure}[h]
	\centering
	\includegraphics[scale=0.8]{17}
	\caption{}
\end{figure}
\begin{LARGE}
we get $\Gamma_{\perp} = \frac{\eta_{2} - \eta_{1}}{\eta_{2} + \eta_{1}}$\\
\end{LARGE}

the reflection coefficent of a line with characteristics impedance  $\eta_{1}$ and terminated at  $\eta_{2}$\\
So the case of normal incidence which is a special case for any oblique incidence at a dielectric interface is equivalent to the transmission line.so when we are dealing with a transmission  line we are handling one of the special case of reflections at the dielectric interface. When  transmission  line is terminated in  $\eta_{21}$,the was getting lost in  $\eta_{2}$ ,now dating we have a dielectric interface,  either we say that the power is in  $\eta_{2}$ or the power is actually not lost at that location but going to the second medium. So the power lost at the dielectric  boundary is the same as the power lost into  $\eta$ as the terminating impedance  of the transmission  line. 
For this normal incidence case with $\theta_{i}$ = $\theta_{t}$ = 0, for parallel polarization
\begin{center}
$\Gamma_{\perp} = \frac{\eta_{2} cos\theta_{i} - \eta_{1} cos\theta_{i}}{\eta_{2} cos\theta_{i} + \eta_{1} cos\theta_{t}}$\newline
\\reduces to
$\Gamma = \frac{\eta_{2} - \eta_{1}}{\eta_{2} + \eta_{1}}$\newline
\\
while\\
$\tau_{\perp} = \frac{2 cos\theta_{i} \eta_{2}}{\eta_{2} cos\theta_{t} + \eta_{1} cos\theta_{i}}$\newline 
\\
$\Gamma_{\perp} = \frac{\eta_{2} -\eta_{1}}{\eta_{2} +\eta_{1}}$\\
hence $\Gamma_{\perp}$ is opposite of $\Gamma_{\parallel}$

\end{center}
$\Gamma_{\perp}$ and $\Gamma_{\parallel}$ having opposite sign is due to the fact that for  $\Gamma_{\parallel}$,$E_{i}$ and $E_{r}$ had same direction. However for  $\Gamma_{\parallel}$,$E_{i}$ and $E_{r}$ had opposite  direction in the tangential component of electric  field that were use to solve for boundary  condition. So the reflection coefficient  in the case of normal incidence is written as $\Gamma = \frac{\eta_{1} - \eta_{2}}{\eta_{2} + \eta_{1}}$  with the assumption that electric field are in the same direction. If $\eta_{1}>\eta_{2}$ you have a phase reversal. If $\eta_{2}>\eta_{1}$ there is no phase reversal. 
So irrespective  of the medium parameter  an angle of incidence, the reflection and transmission  coefficient  are all real quantities.this may be direction reversal for electric and magnetic field,  but there is no arbitrary  phase change either at the transmitted or reflected  wave  .this case is a rather simple reflection and refraction cases. 