\chapter{Impedance Transformation in a lossless medium using a Smith Chart}\label{lec:lec8}
\section{Objectives}
The objective of the chapter is to discuss the following
\begin{enumerate}[(i)]
\item How to develop the Voltage Standing Wave Ratio (VSWR) set of circles.
\item Relationship between impedance and admittance.
\item How to determine reflection coefficient graphically.
\item How to determine impedance transformation graphically.
\item How to locate maximum and minimum voltage, and current points on the transmission line on the Smith Chart.
\end{enumerate}

Previously, we developed a graphical tool by transforming the complex impedance of a transmission line into the gamma plane called the Smith Chart. This we have seen as the superimposition of circles of constant resistance on the circles of constant reactance masked within the limit circle $|\Gamma| \leq 1$. Before we go into the use of the Smith Chart for transmission line calculations, we develop one more set of circles called the constant VSWR circles and then superpose it in the Smith Chart.

We know that:
\begin{equation*}
\Gamma(l) =\Gamma_L e^{-j2\beta{l}}
\end{equation*}
Where $l$ is the distance from load point towards generator, $\Gamma_{L}$ is the voltage reflection coefficient at load end and $\beta$\ is the phase constant of the transmission line\footnote{
We are still discussing lossless transmission lines for which the attenuation constant is zero.
}. The  voltage reflection coefficient, $\Gamma_L$ can be expressed in polar form as $|\Gamma_{L}|e^{j\theta_L}$ such that
\begin{equation*}
\Gamma{(l)}=|\Gamma_{L}|e^{j\theta_L}.e^{-j2\beta l}
\end{equation*}
Where $\theta_L$ is the phase angle of deflection of the load.
\begin{equation}
\Gamma{(l)} =|\Gamma_L|e^{j(\theta_L - 2\beta{l})}
\end{equation}
The total phase becomes $\theta_L - 2\beta{l}$. Hence the reflection coefficient $\Gamma{(l)}$ has magnitude $|\Gamma_L|$ on the complex gamma plane but a phase angle that varies with distance from the load end. Since $l$ is positive when moving towards the generator, therefore, the phase increases in negative direction. The distance of the point $|\Gamma_L|$ from the centre of the circle remains constant but it's angle changes with a change with respect to $l$. Hence $|\Gamma_L|e^{j(\theta_L - 2\beta l)}$ represents a \emph{constant VSWR circle}\index{constant vswr circle} having the same centre of origin with that of the complex gamma plane as shown in figure~\ref{fig:lkjhgryn}.

When moving towards the generator with increasing length, $l$, a corresponding anti-clockwise rotation in the gamma plane is experienced. On this circle, the magnitude of the reflection coefficient is the same no matter what point we are, but the angle differs and we know that,
\begin{equation}
VSWR = \frac{1 + |\Gamma_L|}{1 - |\Gamma_L|}
\end{equation}
\begin{figure}[h]
\centering
\includegraphics[width=0.7\linewidth]{./graphics/lkjhgryn}
\caption{Constant VSWR Circle}
\label{fig:lkjhgryn}
\end{figure}

Since $|\Gamma_L|$ is same at any point on the circle, then the VSWR is same for all points on the circle, hence we call the circle a constant VSWR circle. 
% Note that while solving transmission line problems, we have to draw constant VSWR circles on the Smith Chart so the Smith Chart readily gives the constant resistance and reactance circles in the limit circle of $|\Gamma_L|\leq 1$.
One should always note that the centre of all VSWR circles is same as that of the origin of  the complex gamma plane for all passive loads. Therefore, the magitude of the reflection coefficient is always less or equal to 1. Larger radius of $|\Gamma_L|$ gives more reflection coefficient with lower impedance match and smaller radius of  $|\Gamma_L|$ gives smaller magnitude of reflection coefficient and a better impedance match (see figure~\ref{fig:poiuyfd}).

When we say that we have better impedance matching on the Smith Chart, visually speaking, we mean that the point closest to the centre of the Smith Chart is better for impedance matching because it represents a smaller magnitude of reflection coefficient. 

With the understanding of superimposing the constant VSWR circle on the Smith Chart, we can therefore solve the transmission line problem.
\begin{figure}[h]
\centering
\includegraphics[width=0.57\linewidth]{./graphics/poiuyfd}
\caption{Impedance match with less reflection}
\label{fig:poiuyfd}
\end{figure}

However as we have said, you may have connections in transmission lines which are in the form of parallel connections and we know that from circuit analysis, anywhere we have parallel connections, it is easier to deal with admittances rather than impedances. Before now, we have been discussing load impedances and load characteristics impedance of transmission lines. However, if we have to make parallel connections on the transmission line, we represent the load as admittances and characteristics admittance before we carry out the analysis on the Smith Chart. 

Thus, in this section, we will determine how the Smith Chart is used when we compute in terms of admittances. Since the Smith Chart gives the normalized impedances of the transmission line, the same thing will be derived for admittances so we shall first define the normalized admittance of the transmission line and for that, we would require what is called the \emph{characteristic admittance}\index{characteristics admittance} of the line. The characteristic admittance is given in equation~\eqref{eqn:characteristicsadmit}
\begin{equation}
Y_o= \frac{1}{Z_o}
\end{equation}
Then, every admittance is normalized to this,
\begin{equation*}
\overline{Y} = \frac{Y}{Y_o} 
\end{equation*}
To determine the reflection coefficient in terms of admittance we replace the impedance $Z$ and characteristics impedance are then expressed as the inverse of admittance $\frac{1}{Y}$ and inverse of the characteristics admittance $\frac{1}{Y_o}$ respectively.
\begin{equation*}
\Gamma = \frac{\frac{1}{Y} - \frac{1}{Y_o}}{\frac{1}{Y} + \frac{1}{Y_o}}
\end{equation*}
The negative sign indicates a phase change of 180\textdegree\footnote{
As -1 is a phase change of $\pi, e^{j\pi} = \cos\pi + j\sin\pi = -1$
} thus, it can be expressed as
\begin{equation}
\Gamma= \frac{Y_o - Y}{Y_o + Y} 
\end{equation}
So the reflection coefficient written in terms of normalized admittances is same as reflection coefficient written in terms of normalized impedance except for the 180\textdegree\; phase change.

This implies that same value is derived for reflection coefficient when calculated using the normalized impedance or admittance except there is a phase shift of 180\textdegree\; that would be experienced on the transmission line. In other words, on the complex gamma plane of the Smith Chart, the 180\textdegree\; phase shift will correspond to a rotation of 180\textdegree\; (see figure~\ref{fig:zstyuiou}). Essentially, the normalized impedance and admittance can be calculated in the same way except when carry out calculations for normalized admittances, there is a rotation of 180\textdegree\; on the complex gamma plane otherwise all other values and parameters remain unchanged.
\begin{figure}[h]
\centering
\includegraphics[width=0.6\linewidth]{./graphics/zstyuiou}
\caption{Normalized impedance point}
\label{fig:zstyuiou}
\end{figure}

Hence as far as the constant VSWR circles are concerned, any point on the Smith Chart rotated by 180\textdegree\; corresponds to the normalized admittance with corresponding circles of constant resistance and circles of constant reactance. 

\emph{What would be the complex representation of the admittance?}
\begin{figure}[h]
\centering

Alternatively, we can keep the Smith Chart orientation the same and rotate the complex gamma axis by 180\textdegree\; before doing calculations for admittances. So, if we develop an understanding that we will not rotate the Smith Chart, that means we want to use it in its original form, that is, the most clustered part of it to our right, then if we do the impedance calculations, the positive real axis is towards right and the positive imaginary axis goes up. However, during calculations using the Smith Chart unchanged for admittances, the positive real axis is towards the left and the  positive imaginary axis will be downwards. Normally, whenever we do the Smith Chart calculations, we do not rotate the Smith Chart. For the impedance, the gamma axis is the  positive top right plane and for admittances, the bottom left plane is the positive plane. Depending on whether we are doing calculations for the impedance or admittances and if we require fixed measurement in the complex gamma plane, then the axis has to be rotated appropriately by 180\textdegree\; depending on whether we are using the impedance or admittance. This is important when finding the phase of the reflection coefficient, otherwise, the axis of the gamma plane is insignificant. So without worrying about the axis of the complex gamma plane, we can use the same Smith Chart for the admittance as well as the impedance calculations. This is the reason why when you look at the Smith Chart carefully, you will see that the upper half of the Smith Chart is denoted by $(x,b)$ and the lower part $(-x,-b)$.The circles are denoted by either $r$ or $g$ so any normalized value of $r$ is equal to the same normalized value of $g$ which represent the same circle. 

Hence, as long we are dealing with a normalized quantity, the impedance and admittance can be treated exactly the same way on the Smith Chart. However, the normalized values of $g$ and $r$ or $b$ and $x$ have different meaning physically. But do they represent the same physical conditions? The answer is NO! For example $r = 0; x = 0$, corresponds to a short circuit condition\textemdash\;the impedance is zero\textemdash\;however, if we take  a normalized value of admittance with $g=0, b=0$ which represents admittance equal to zero, it is not a short circuit but an open circuit condition on the line. Therefore, the normalized values of impedances and admittances can be treated exactly the same way in calculations but not the same interpretations for the physical conditions. In summary, mathematically, $g = 0\quad, b = 0\Longrightarrow$ open circuit and $g = \infty, b = \infty \Longrightarrow$  short circuit. 
\begin{figure}[h]
\centering
\includegraphics[width=1\linewidth]{./graphics/jnbkvfld}
\caption{Physical variation of impedance and admittance}
\label{fig:jnbkvfld}
\end{figure}

Let us therefore consider the points on Smith Chart for the admittance where the special points $r$ and $x$ are replaced with $b$ and $g$ respectively. Maintaning the original orientation of the Smith Chart then for the admittance Smith Chart, the upper half of the Smith Chart represents capacitive loads and the lower half represents the inductive loads (see figure~\ref{fig:jnbkvfld}). With these in mind, the use of Smith Chart for impedance or admittance calculation is very straight forward.

Now lets make use of the Smith Chart to solve transmission line problems. 

\section{Use of Smith Chart for Transmission Line Calculation}
Let us consider the simplest problem we can think of for transmission line. Suppose for a given load, we want to find the reflection coefficient at the load point. Analytically, we can use the formula $\frac{Z_L - Z_0}{Z_L + Z_0}$ but in using the Smith Chart, the problem is much simpler to solve. Remember the impedance and admittance you have on Smith Chart are all normalized quantities. So we take the following steps

\subsection*{Procedure:}
\begin{enumerate}[(i)]
\item Normalize $Z_{L}$ to get $\bar{Z}_{L}$, that is, $\frac{Z_{L}}{Z_{0}}$.
\item Determine this point on the Smith Chart, that is,  $r+jx$ by identifying the constant resistance and reactance circle with the value $r$ and $x$ respectively and the point of intersection (see figure~\ref{fig:lfds}).
\begin{figure}[h]
\centering
\includegraphics[width=0.7\linewidth]{./graphics/lfds}
\caption{Point of intersection of circles of constant resistance and reactance}
\label{fig:lfds}
\end{figure}

\item Determine the distance of the point of intersection from the origin to find the magnitude of the reflection coefficient, $|\Gamma_L|$ and then the corresponding phase given as $\theta_L$.
\end{enumerate}

So without doing any calculation, just by measuring its distance, and the angle, we get the magnitude of reflection coefficient, $|\Gamma_L|$
and phase angle, $\theta_L$. 

Similarly, for the admittance, let $\bar{Y} = g + jb$. We determine the point of intersection on the Smith Chart the same way as that of impedance, however, to find out the complex reflection coefficient, maintaining the orientation, then the positive real axis is to the left. So the phase angle will be measured from the positive real axis as shown in figure~\ref{fig:kjhgfds}. We summarize the steps as follows.
\begin{figure}[h]
\centering
\includegraphics[width=0.7\linewidth]{./graphics/KJHGFDS}
\caption{Determining the reflection coefficient for admittance on a Smith Chart}
\label{fig:kjhgfds}
\end{figure}
 
\subsection*{Procedure:}
\begin{enumerate}[(i)]
\item Normalize the admittance $\bar{Y}$ with the characteristic admittance, $Y_o$.
\item Determine the point on the Smith Chart which is the point of intersection of the circle of constant conductance, $g$ and the circle of constant susceptance, $b$.
\item Determine the distance of the point from origin to find the magnitude of the reflection coefficient,  $|\Gamma_L|$ but the phase angle, $\theta_L$ will be measured from the  positive real axis (see figure~\ref{fig:kjhgfds}). 
\end{enumerate}

Thus, when we are using normalized impedance or admittance, appropriate rotation has to be made on the coordinate axes of the Smith Chart. Once this is done, the calculation of the complex reflection coefficient is very straight forward. Mark the normalized points on the Smith Chart. Find the distance from the origin to the marked point and then measure the angle from this marked point from the positive real axis which differs for impedance and admittance while maintaining the orientation of the Smith Chart. 

In another simple transmission line problem, let us suppose the magnitude of the reflection coefficient and phase angle is given, but we need to find the load. The steps are summarized below.
\subsection*{Procedure:}
\begin{enumerate}[(i)]
\item Draw a cirle of radius equal to the magnitude of the reflection coefficient from the origin.
\item Determine phase angle from the positive real axis depending on whether the impedance or admittance value at the load is required.
\item Mark the point on the circle and find the corresponding $r$ and $x$ circles or $g$ and $b$ circles.
\item Multiply the normalized impedance or admittance by the characteristic impedance or admittance to get the load impedance or admittance.
\end{enumerate}

\section{Impedance transformation problems with the Smith Chart}
A common problem in transmission line is impedance transformation. We have shown how it is done analytically but \emph{how can we perform impedance transformation graphically?} 
\begin{figure}[h]
\centering
\includegraphics[width=0.7\linewidth]{./graphics/wertuyuk}
\caption{Tranformed impedance at distance $l$ from the load}
\label{fig:wertuyuk}
\end{figure}

If the load impedance is given as shown in figure~\ref{fig:wertuyuk} and we are asked to find the reflection coefficient and load impedance at another point $l$ on the transmission line, that is, given $Z_L$ or $Y_L$, we are to find $Z_L$ or $Y_L$ at distance $l$ from the load. We will use a hypothetical sketch of the Smith Chart shown in figure~\ref{fig:uytrewsxcvbj}.
\begin{figure}[h]
\centering
\includegraphics[width=0.9\linewidth]{./graphics/uytrewsxcvbj}
\caption{Points along the constant r and x Circles}
\label{fig:uytrewsxcvbj}
\end{figure}

\subsection*{Procedure:}
\begin{enumerate}[(i)]
\item If the impedance is not normalized, we normalize all impedances.
\item Determine the point, $\bar{Z}_L$, on the Smith Chart and draw a circle passing through the point from the origin. As we have seen earlier, the magnitude of the reflection coefficient remains same as the point moves on the circle which is a constant VSWR circle.
\item Determine the point that translates to a distance $l$. In a Smith Chart, one rotation, $2\pi$ translate to a phase change of $2\beta{l}$\footnote{
As we move along the transmission line, the phase angle changes by $2\beta{l}$ in the clockwise direction (from load towards generator).
}, that is $2\beta{l} = 2\pi\Longrightarrow l = \frac{\pi}{\beta} = \frac{\lambda}{2}$\footnote{
$\beta = \frac{2\pi}{\lambda}$
} (see figure~\ref{fig:mjhtre}). Therefore, a full rotation around the Smith Chart corresponds to a distance of $\frac{\lambda}{2}$ on a constant VSWR circle\footnote{
The above analysis makes sense since one characteristics of the transmission line is that it's impedance characteristics repeats every $\frac{\lambda}{2}$ distance on the line.
}. So we determine the corresponding $x\lambda$ for the value of $l$ moving clockwise along the VSWR from the point $\bar{Z}_L$.
\begin{figure}[h]
\centering
\includegraphics[width=0.7\linewidth]{./graphics/mjhtre}
\caption{Tranformed impedance at distance $l$ from the load}
\label{fig:mjhtre}
\end{figure}

\item Mark the new point on the Smith Chart and determine the magitude of the reflection coefficient and phase angle as shown in figure~\ref{fig:uyhbgjvkclxse} or the circle of constant resistance and circle of constant reactance to determine the normalized impedance at $l$, that is, $\bar{Z}_{l} = r + jx$.
\begin{figure}[h]
\centering
\includegraphics[width=0.7\linewidth]{./graphics/uyhbgjvkclxse}
\caption{New phase angle after rotation}
\label{fig:uyhbgjvkclxse}
\end{figure}

\item Multiply the normalized impedance at $l$ by the characteristic impedance $Z_o$ to get $\bar{Z}_{l}$ at the new location.
\end{enumerate}

For admittance calculations, the same steps apply except in the sense of the position of the positive real axis (see figure~\ref{fig:dfyui}).
\begin{figure}[h]
\centering
\includegraphics[width=0.7\linewidth]{./graphics/dfyui}
\caption{Phase angle measurement with admittance Smith Chart}
\label{fig:dfyui}
\end{figure}


\section{VSWR on the transmission line}

We want to find out the distance of current or voltage maximum at the load end of the transmission line.
\begin{figure}[h]
\centering
\includegraphics[width=0.7\linewidth]{./graphics/oijhgfdsa}
\caption{Maximum and Minimum Points of Resistance}
\label{fig:oijhgfdsa}
\end{figure}

From our knowledge of transmission line, at point of maximum voltage, we experience minimum current which gives $R_{max}$ then at minimum voltage, we experience maximum current which gives $R_{min}$ hence $R_{max}$ corresponds to $V_{max}$,  $I_{min}$ and  $R_{min}$ corresponds to  $V_{min}$, $I_{max}$.  We want to find out these locations on the load. To move from point $Z_{l}$ to $\bar{R}_{max}$, the clockwise angle covered indicates movement $l$ towards the generator corresponding to $\theta_{max}$ while $\theta_{min}$ corresponds to $180^o$ or $\frac{\lambda}{4}$ distance to get to $R_{min}$ from the $R_{max}$ point.
\begin{figure}[h]
\centering
\includegraphics[width=0.6\linewidth]{./graphics/lkjtresx}
\caption{Maximum and Minimum Value of Voltage, Current, and Resistance}
\label{fig:lkjtresx}
\end{figure}

$\frac{\theta_{max}}{2\beta}=l$ gives the distance from the load point $\theta_{max} < 180^o$. From the diagram, $\theta_{min}$ indicates that the angle moved to get to $R_{min}$ is $>270^o but <360^o$. $\theta_{max}$ is therefore the angle moved to get from load to $R_{max}$.
\begin{align}
\frac{\theta_{min}}{2\beta}=\frac{\theta_{max} + \pi}{2\beta} = L
\end{align}
To get to $R_{min}$ position; $\theta_{min} = \frac{\theta_{max}}{2\beta} + \frac{\lambda}{4}.$ 