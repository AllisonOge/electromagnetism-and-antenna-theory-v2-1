\chapter{Impedance Transformation.}
In our previous chapter, we investigated the Standing Wave Pattern on a transmission line and also an important parameter which is the voltage standing wave ratio VSWR, (which is the ratio of the maximum voltage seen on the transmission line to the minimum voltage seen on the transmission line).

In this chapter, we will study the impedance transformation on a lossless
transmission line and we will establish some of the very important characteristics of impedance transformation on a lossless transmission line. After which, we will proceed to an important calculation of Power Transfer to the Load and also the expression for $V^{+}$.
\section{Transformation on a Lossless Transmission Line}
Recall that, the impedance transformation relationship for any point is given as :
\begin{align*}
\footnotemark[1]Z(l) = Z_o\left\lbrace \frac{Z_Lcosh\gamma l + Z_osinh\gamma l}{Z_ocosh\gamma l + Z_Lsinh\gamma l}\right\rbrace 
\end{align*}
\footnotetext[1]{$Z(l)$ is the impedance at any point on the line}
\footnotetext[2]{$\bar{Z}(l)$ is the normalized impedance at any point on the line}
\begin{align*}
\footnotemark[2]\bar{Z}(l) = \left\lbrace \frac{\bar{Z}_Lcosh\gamma l + sinh\gamma l}{\bar{Z}_Lsinh\gamma l + cosh\gamma l}\right\rbrace 
\end{align*}
if $\bar{Z_L} = 1$ i.e $Z_L = Z_o$ and $\bar{Z}(l) = 1 \rightarrow  Z(l) = Z_o$\\
for a lossless Transmission Line,$\gamma=j\beta$\\\\
Where
\begin{align*}
cosh\gamma l = \frac{e^{\gamma l} + e^{-\gamma l}}{2} \quad and \quad sinh\gamma l = \frac{e^{\gamma l} - e^{-\gamma l}}{2}
\end{align*}
so,
\begin{align*}
cosh\gamma l= cosh(j\beta l)=\frac{e^{j \beta l} + e^{-j \beta l}}{2}=cos\beta l
\end{align*}
And
\begin{dmath*}
sinh\gamma l=sinh(j \beta l) = \frac{e^{j \beta l} - e^{-j \beta l}}{2}=j\left( \frac{e^{j \beta l} - e^{-j \beta l}}{2j}\right) =jsin\beta l
\end{dmath*}

Therefore impedance transformation relationship for a lossless transmission line is given as;
\begin{equation}
\bar{Z}(l) = Z_{o}\left\lbrace \frac{\bar{Z}_Lcos\beta l + jsin\beta l}{cos\beta l + j\bar{Z}_Lsin\beta l}\right\rbrace 
\end{equation}
$Z_o$ is a real quantity for a lossless transmission line same as:
\begin{equation}
\bar{Z}(l) = \left\lbrace \frac{\bar{Z}_Lcos\beta l + jsin\beta l}{cos\beta l + j\bar{Z}_Lsin\beta l}\right\rbrace 
\end{equation}
With this impedance transformation relationship, we can establish some very important characteristics of the transmission line. When we move on a transmission line a distance of $\frac{\lambda}{2}$ ( half wave length ), the voltage standing wave characteristics repeat itself, so if we take a ratio of voltage and current at a certain location, we expect that this characteristic will repeat at $\frac{\lambda}{2}$.\\
Moving $\frac{\lambda}{4}$ from the point of $R_{max}$ and $R_{min}$ , should bring something interesting. Similarly, if we terminate the line into its characteristics impedance, the impedance measured at any point equals to the characteristics impedance. So we have three important points we can draw from this.\\
Lets show with proof these three characteristics.
\begin{enumerate}[(i)]
\item Impedance Value repeat every $\frac{\lambda}{2}$ distance: Lets say at $l$ we have maximum or minimum impedance, then moving $l=\frac{\lambda}{2}$ from that point we should get same impedance again!\\\\
At location $l , Impedance =\bar{Z}(l),$
\begin{figure}[h]
\centering
\includegraphics[width=0.7\linewidth]{./graphics/astyui;f}
\caption{Z($l$) over distance $\frac{\lambda}{2}$}
\label{fig:astyuif}
\end{figure}

At location ${(l+\frac{\lambda}{2})}$, Impedance  $=$ $\bar{Z}(l+\frac{\lambda}{2})$\\\\
substituting ${(l+\frac{\lambda}{2})}$ for $l$ in equation $6.1$, we get: 
\begin{align*}
\bar{Z}(l+\frac{\lambda}{2}) = \left\lbrace \frac{\bar{Z}(l)cos\beta (l+\frac{\lambda}{2}) + jsin\beta (l+\frac{\lambda}{2})}{cos\beta (l+\frac{\lambda}{2}) + j\bar{Z}(l)sin\beta (l+\frac{\lambda}{2})}\right\rbrace 
\end{align*}
Since \footnote{$\beta$ represents phase constant}$\beta$ = $ \frac{2\pi}{\lambda}$, therefore,
\begin{dmath*}
\beta(l+\frac{\lambda}{2})=\frac{2\pi}{\lambda}(l+\frac{\lambda}{2})=\frac{2\pi}{\lambda}l+\pi=\beta l+\pi
\end{dmath*}
\begin{align*}
\bar{Z}(l+\frac{\lambda}{2}) = \left\lbrace \frac{\bar{Z}(l)cos(\beta l+\pi) + jsin(\beta l+\pi)}{cos(\beta l+\pi) + j\bar{Z}(l)sin(\beta l+\pi)}\right\rbrace 
\end{align*}
But 
\begin{align*} 
cos(\beta l+\pi)=-cos\beta l \quad and \quad sin(\beta l+\pi)=-sin\beta l
\end{align*}
\begin{dmath*}
\bar{Z}(l+\frac{\lambda}{2})=\left\lbrace \frac{-\bar{Z}(l)cos\beta l - jsin\beta l}{-cos\beta l - j\bar{Z}(l)sin\beta l}\right\rbrace = \left\lbrace \frac{\bar{Z}(l)cos\beta l + jsin\beta l}{cos\beta l + j\bar{Z}(l)sin\beta l}\right\rbrace\;\;\equiv \bar{Z}(l)
\end{dmath*} 
This is same as original formula obtained for $\bar{Z}(l)$.

Hence\ $\bar{Z}(l+\frac{\lambda}{2})
=\bar{Z}(l)$, which proves that impedance repeat itself at $\frac{\lambda}{2}$.\\

In other words, no matter the length of the transmission line, modulus $\frac{\lambda}{2}$ is  a special information which is available from the impedance relationship.
\item If we move a distance of $\frac{\lambda}{4}$, we move from a maximum to minimum impedance and vice versa.\\

At location $l , Impedance =\bar{Z}(l).$\\\\
Then at location ${(l+\frac{\lambda}{4})}$, Impedance  $=$ $\bar{Z}(l+\frac{\lambda}{4})$\\\\
substituting ${(l+\frac{\lambda}{4})}$ for $l$ in equation $6.1$, we get:
\begin{align*}
\bar{Z}(l+\frac{\lambda}{4}) = \left\lbrace \frac{\bar{Z}(l)cos\beta (l+\frac{\lambda}{4}) + jsin\beta (l+\frac{\lambda}{4})}{cos\beta (l+\frac{\lambda}{4}) + j\bar{Z}(l)sin\beta (l+\frac{\lambda}{4})}\right\rbrace 
\end{align*}
Where\ \ \  $\beta( l + \frac{\lambda}{4})$ = $\beta l + \frac{2\pi}{\lambda} . \frac{\lambda}{4} = \beta l + \frac{\pi}{2}.$
\begin{align*} 
= \left\lbrace \frac{\bar{Z}(l)cos(\beta l + \frac{\pi}{2}) + jsin(\beta l + \frac{\pi}{2})}{cos (\beta l + \frac{\pi}{2}) + j\bar{Z}(l)sin(\beta l + \frac{\pi}{2})}\right\rbrace
\end{align*}

But $cos(\beta l + \frac{\pi}{2})= -sin\beta l , \quad sin(\beta l +\frac{\pi}{2})= cos\beta l$

\begin{align*} 
= \left\lbrace \frac{-\bar{Z}(l)sin\beta l + jcos\beta l}{-sin\beta l + j\bar{Z}(l) cos\beta l}\right\rbrace
\end{align*}

\begin{dmath*}
\bar{Z}(l+\frac{\lambda}{4}) =\frac{j}{j} \left\lbrace \frac{cos\beta l + j\bar{Z}(l)sin\beta l}{\bar{Z}(l)cos\beta l + jsin\beta l}\right\rbrace = 
\frac{1}{\left\lbrace \frac{\bar{Z}(l)cos\beta l + jsin\beta l}{cos\beta l + j\bar{Z}(l)sin\beta l}\right\rbrace} =\frac{1}{\bar{Z} (l)}
\end{dmath*}
\begin{align*}
\bar{Z}(l+\frac{\lambda}{4}) = \frac{1}{\bar{Z} (l)}
\end{align*}
This is another important point here, that for every $\frac{\lambda}{4}$ distance moved , the normalized impedance inverts itself. Note the word \textbf{Normalized}, it is not absolute impedance. If absolute impedance is reversed, it becomes \textbf{Admittance}. The normalized impedance does not have a unit, it is dimensionless. So if I have an impedance greater than $Z_o$ along a transmission line , after $\frac{\lambda}{4}$ distance, it becomes less than $Z_o$ because the normalized impedance is the inverse of the one at previous location. So at $\frac{\lambda}{4}$ distance, impedance inverts after another $\frac{\lambda}{4}$ again it re-inverts becoming the original impedance, which is more like repeating itself at $\frac{\lambda}{2}$ distance (Which was proved earlier). So when we talk about the periodicity of the impedance on the transmission line, at $\frac{\lambda}{2}$ the absolute or normalized impedance repeats itself, whereas every distance of $\frac{\lambda}{4}$, the normalized impedance inverts itself. Later on when we study impedance matching characteristics, this property is used extensively for finding out the impedance transformation which can match impedance on transmission line.
\item The matching condition characteristics: If the line is terminated in its characteristics impedance, then the impedance seen at every point on the transmission line is equal to the characteristics impedance. So if $Z_L=Z_o$ $\Longrightarrow \bar{Z}_L= \frac{Z_L}{Z_o}$=1. Then equation $6.1$ becomes;
\begin{dmath*}
\bar{Z}(l) = \left\lbrace \frac{\bar{Z}_Lcos\beta l + jsin\beta l}{cos\beta l + j\bar{Z}_Lsin\beta l}\right\rbrace = \left\lbrace \frac{cos\beta l + jsin\beta l}{cos\beta l + jsin\beta l}\right\rbrace = 1
\end{dmath*}
Therefore, irrespective of the length of the transmission line, if the line is terminated in its characteristics impedance , then the impedance seen at every point on the transmission line is equal to the characteristics impedance. This means that once a line is terminated in its characteristics impedance, we don't have to worry about the impedance transformation on the line, we can use any length of transmission and the impedance will always be the same along the length of the line. $Z_L=Z_o$ i.e when reflection coefficient is zero, there is no reflected wave on the transmission line and we have only forward traveling wave on the transmission line. Forward traveling wave always see an impedance which is equal to the characteristics impedance. This result is not new, it is what we had discussed earlier when we talked about transmission line and that was, if a line is terminated in its characteristic impedance , the impedance seen at every point on the transmission line is equal to the characteristics impedance. 
\end{enumerate}
So these are the three very important characteristics of a lossless transmission line.
\begin{enumerate}[(i)]
\item Impedance transformation repeats at every $\frac{\lambda}{2}$ distances.
\item Normalized impedance inverts at every $\frac{\lambda}{4}$ distance.
\item Finally if the line is terminated in its characteristics impedance , the impedance seen at every point of the line is equal to the characteristics impedance.
\end{enumerate}
With this understanding of impedance transformation, now we can go to Power Transfer Calculation of the transmission line.

\section{Power Transfer on Transmission Line}
Initially, our idea was to transfer power from generator to load effectively. In the lossless case, it should be that all generator power is completely transferred to the load. However, we have seen that if the impedance is not equal to the characteristics impedance, then there will always be reflection on the transmission line and whatever energy the generator supplies, part of this energy will get reflected back to the generator . Now when we talk about matching condition or maximum power transfer condition, there are two cases to consider;
\begin{enumerate}[(i)]
\item When the power is given by the generator, it should be maximally transfered to the load.
\item When the reflected power comes back to generator, the generator is not capable of absorbing power, so when the reflected power comes back with a different amplitude and phase, it negatively affects the generator's performance. So it is desirable that the generator should not see any power coming back at it.
\end{enumerate}
For these two reasons that the generator power should be completely delivered to the load and that no reflected power should come back to the generator, we must make sure always that the impedance which the generator sees is always equal to the characteristics impedance. We will study these two issues later but let's study a general case and ask if we have a transmission line which is connected to a generator at  one end and a load at the other end. How much power will be delivered to the load. Again without going into voltage and current equation, we can write down power at the location of the load i.e Power delivered to the load will be given as follows.

At load end, $L=0$ therefore, $e^{-2\beta (l)} = e^{-2\beta (0)}$

$V(o)= V^{+} \left\lbrace {1 + \Gamma_L e^{-2\beta(0)}}\right\rbrace $ = $V^{+}\left\lbrace 1 +\Gamma_L \right\rbrace$.

$I(o)= \frac{V^{+}}{Z_o} \left\lbrace {1 - \Gamma_L e^{-2\beta(0)}}\right\rbrace $ =$ \frac{V^{+}}{Z_o}$ $\left\lbrace 1 -\Gamma_L \right\rbrace$ 

So for the general voltage and current relationship on the transmission line we have found  out the voltage and current values at load end.

$Z_o= $ real for a lossless line,
\begin{align*}
I^\ast (o) =\frac{V^{+ (\ast )}}{Z_o}\left\lbrace 1 -\Gamma_L \right\rbrace
\end{align*}

Power delivered to load.
\begin{dmath*}
P= \frac{1}{2} Re\left\lbrace V(o) I^*(o) \right\rbrace = \frac{1}{2} Re\left\lbrace V^+(1+\Gamma_L) \times \frac{V^{+(*)}}{Z_o} (1-\Gamma_L)\right\rbrace
\end{dmath*}
\begin{equation}
P= \frac{1}{2} \frac{|V^+|^2}{Z_o} \left\lbrace 1 -|\Gamma_L|^2 \right\rbrace 
\end{equation}

$\Gamma_L$ is a real value since it is the absolute value taken here.

\textbf{NOTE:} 

$V= a + jb  \quad \quad V^*= a -jb$

$ V\times V^* = ( a + jb)(a-jb)= a^2 - jab + jab + (jb) (-jb)= a^2 + b^2 $

$| V |= \sqrt{a^2 + b^2} $ \quad \quad $|V|^2 = a^2 + b^2 $

$| V |^2 = V \times V^* $ 

Recall that,
\begin{align*}\Gamma_L = \frac{ Z_L -Z_o }{ Z_L + Z_o }.
\end{align*}

Once load impedance is known, the reflection coefficient at load end is known. Then we can calculate modulus of reflection coefficient so that the power delivered to the load can be calculated if $V^+$ is known. How do we find $V^+ ? $ From the relationship here, if we know the amplitude of the forward traveling wave as well as the load impedance, we can calculate the power from the circuit perspective point of view. We can use a different argument in arriving at same answer. That is on the transmission line the power supplied to the generator, in the form of a traveling wave goes towards the load and we already said that traveling wave always see impedance equal to characteristics impedance. So if the traveling wave has amplitude $V^+$, it is as if this wave is supplying the power to $Z_o$ in the lossless case. So one can say now that  a wave having amplitude $V^+$ going forward direction sees impedance $Z_o$.\\
Power carried by forward wave:  
\begin{align*}
P_{for}= \frac{1}{2} \frac{{| V |^+}^2}{Z_o}
\end{align*}
When this gets to the load , part of the energy will get reflected back , and this backward traveling wave has amplitude $V^-$ which also sees characteristics impedance.\\
The power carried by this wave:
\begin{align*}
P_{ref}= \frac{1}{2}\frac{{|V^-|}^2}{Z_o}
\end{align*}
which is the power reflected by load in the backward wave.

Therefore, Power delivered to load $=$ Power transferred to load $-$ Power reflected.

\begin{align*} 
P= \frac{1}{2} \frac{ { |V^+ | }^2}{Z_o} -\frac{1}{2} \frac{ {|V^- |}^2 }{Z_o} = \frac{1}{2} \left\lbrace \frac{ { | V^+ | }^2}{Z_o} -\frac{ {|V^- |}^2 }{Z_o} \right\rbrace
\end{align*}
\begin{align*} 
P= \frac{1}{2} \frac{ { | V^+ | } ^2}{Z_o} \left\lbrace 1 - { | {\frac{V^-}{V^+}|^2}}\right\rbrace\\
Recall \ that\ \ \ \  \Gamma_L =\frac{V^-}{V^+}.
\end{align*}
\begin{align*}
P=\frac{1}{2} \frac{{| V^+ |}^2}{Z_o} \left\lbrace 1 - { \Gamma_L }^2 \right\rbrace.
\end{align*}
So when we do a power calculation for transmission line , either we go by circuit concept or we go by the understanding of the wave concept. This is the story of the real part i.e the power actually supplied to the load. One then asks how do we calculate for complex power at any location $?$. If we calculate the power flow along any point on the transmission line not necessarily at the load end, what will that indicate $?$ Lets see, we can get the voltage and current at any arbitrary location on the transmission line i.e
\begin{align*} 
V(l) = V^+ e^{j\beta l} \left\lbrace 1 + \Gamma_L e^{-j2\beta l} \right\rbrace ,\\ 
I(l) = \frac{V^+}{Z_o} e^{j\beta l} \left\lbrace 1 - \Gamma_L e^{-j2\beta l} \right\rbrace
\end{align*}
Complex Power at Location $l$ is ;
\begin{align*}
P = \frac{1}{2} (V I^*) =
\end{align*}
\begin{align*}
\frac{1}{2} [V^+ e^{j\beta l} \left\lbrace 1 + \Gamma_L e^{-j2\beta l} \right\rbrace ] \times [\frac{V^+}{Z_o} e^{-j\beta l} \left\lbrace 1 - \Gamma_L e^{j2\beta l} \right\rbrace].
\end{align*}
\begin{align*}
=\frac{1}{2} \frac{{| V ^+ |}^2}{Z_o} [( 1 + \Gamma_L e^{-j2\beta l})(1 - \Gamma e^{j2\beta l})]
\end{align*}
\begin{align*}
= \frac{1}{2} \frac{{ | V^+ | }^2}{Z_o} [ 1 - \Gamma_L e^{j2\beta l} + \Gamma_L e^ {-j2\beta l} - {| \Gamma_L | ^2}]
\end{align*}
\begin{align*} 
=\frac{1}{2} \frac{{ | V^+ | }^2}{Z_o} ( 1 - {|\Gamma_L}|^2 -\frac{2j(\Gamma_L e^{j2\beta l} - \Gamma_L e^{-j2\beta l})}{2j} )
\end{align*}
\begin{align*} 
=\frac{1}{2} \frac{{ | V^+ | }^2}{Z_o} ( 1 - {|\Gamma_L}|^2 -2j(\Gamma_L \frac{ ( e^{j2\beta l} - e^{-j2\beta l})}{2j}) 
\end{align*}
\begin{align*} 
=\frac{1}{2} \frac{{ | V^+ | }^2}{Z_o} ( 1 - {|\Gamma_L}|^2 -2j\Gamma_L sin2\beta l ) 
\end{align*}
\begin{equation} 
=\frac{1}{2} \frac{{ | V^+ | }^2}{Z_o}  [1 - {|\Gamma_L}|^2 + j2I_m(\Gamma_Le^{-j2\beta l})]
\end{equation}
Real Part $=$ $1 - {|\Gamma_L}|^2$ $=$ Resistive Power\\
Imaginary Part $=$ $ j2I_m(\Gamma_Le^{-j2\beta l})$ $=$ Reactive Power\\\\
\textbf{NOTE} ; The real power is the power delivered to the location.\\   
Therefore, at any location of the transmission line , the power is complex , the interesting thing to note is that the resistive power at any point is equal to the power which we earlier calculated was delivered to the load end. So for a Lossless line the resistive power at any point along the line is same as that which was delivered to the load . Since  this resistive power finally gets to the load if the line is lossless, since there is no absorption of power at any point along the line and so all the resistive power gets to the load because the load is the part where you have resistive component and power can be  absorbed in that location . So at any point on transmission line, we see power flow and the resistive part ultimately gets delivered to the load.  So the resistive power which is actual power flow should be independent of location on the line.

However the reactive power is a function of $l$, which tells the energy stored at different location along the transmission line. Now we have two parts when we calculate the power on a transmission line . There is a resistive power which is a measure of power flow which ultimately gets delivered to the load. Reactive power tells you the amount energy storage at different location along the line which depends on the value of voltage and current at that location. Because of standing wave , the voltage and current varies at that location , and also the energy storage also varies at different location along the transmission line. In general, the reactive power will vary at different locations across the line, the resistive power which is the power delivered to the load will be independent of location of the transmission line.\\
Let us come to the final question on transmission line analysis, everything we have done so far, impedance transformation relationship, power transfer analysis and so on, we have not obtained the absolute value of $V^+$, as we kept avoiding its absolute value by working in terms of $l$ or $\Gamma$ that was an implicit ratio of $V^+$ somehow. $V^+$ is a final arbitrary constant in the solution of the differential equation of transmission line that we are yet to evaluate. Now for power flow calculation of the transmission line, for the first time we require the absolute value of $V^+$, we cannot talk about relative quantities anymore.

\section{Evaluation of \textbf{$ V ^ {+} $} } 
Considering the Circuit below, $Z_L$ is transformed to \footnote{$Z^{'}_L$ newly transformed impedance}$Z^{'}_L$, from position $BB^{'}$ to $AA^{'}$.\\
$Z^{'}_L$ is the impedance of the load as seen by the generator end. With $Z_L$ transformed to $Z^{'}_L$ , the whole circuit is reduced to a lump circuit as shown in the figure below.
\begin{figure}[h]
\centering
\includegraphics[width=0.5\linewidth]{./graphics/qwerrtt}
\caption{Transformation from $Z_L$ to $Z_{L}^'$}
\label{fig:qwerrtt}
\end{figure}

\begin{equation}
V_A = \frac{Z^'_L}{Z^{'}_L + Z_s} . V_s
\end{equation}
\begin{equation}
I_A = \frac{V_A}{Z^{'}_L}
\end{equation} 
So from the lumped circuit at generator end we can determine the voltage and current at the input end of the transmission line. We can write down the voltage and current at the input end using the transmission line equations.
\begin{equation*} 
V_A= V(l) = V^+ e^{j\beta l} \left\lbrace 1 + \Gamma_L e^{-j2\beta l} \right\rbrace 
\end{equation*} 
\begin{equation*}
I_A = I(l) = \frac{V^+}{Z_o} e^{j\beta l} \left\lbrace 1 - \Gamma_L e^{-j2\beta l} \right\rbrace
\end{equation*}

So we know now the value of $V_A$ and $I_A$ from the lumped element side of view and the transmission line element side of view. We can equate the two to be same to get,
\begin{equation} 
V_A = \frac{Z^{'}_L}{Z^{'}_L + Z_s} . V_s = V^+ e^{j\beta l} \left\lbrace 1 + \Gamma_L e^{-j2\beta l} \right\rbrace 
\end{equation}
\begin{equation}
I_A = \frac{V_A}{Z^{'}_L} = I(l) = \frac{V^+}{Z_o} e^{j\beta l} \left\lbrace 1 - \Gamma_L e^{-j2\beta l} \right\rbrace
\end{equation}
So solving equation $6.7$ making $V^{+}$ the subject of the formula, we get the expression of $V^+$ below. 
\begin{equation} 
V^+ = \frac{Z^{'}_L V_s e^{-j\beta l}}{(Z_S + Z^{'}_L)(1 + \Gamma_L e^{-j2\beta l })}, 
\end{equation}
$\alpha, \beta , l  , \Gamma_L,$ and $Z_o$ are all known values, so now we have a known value for $V^+$\\

From here $V^+$ can be determined and then substituted into the power equation and so power delivered to the load can be calculated or power at any point along the line. This is a complete solution to voltage and current on the transmission line.

\section{Summary}  
We defined a very important parameters of transmission line called the reflection coefficients and the voltage standing wave ratio ( VSWR ) . Then we studied the impedance transformation relationship on a lossless transmission line. We took special case of lossless transmission line and then we found some important characteristics of the lossless transmission line. We got the differential equation for voltage and current. We solved the voltage and current equations and then got a general solution for voltage and current on the transmission line. Then we imposed a boundary condition that was the impedance boundary condition on the transmission line and then from there we evaluated certain arbitrary constants on the transmission line. Furthermore, we investigated the power flow on transmission line, found out how much power would be delivered to the load and ultimately we found out the final unknown arbitrary constant which was $V^+$ on transmission line so that we can now calculate the absolute power delivered to the load for a given condition. Again, to the concept of lumped element, increase in frequency is not applicable,  this is because, space has to be brought into the picture and we introduced the concept called the \textbf{Distributed Element}. In the framework of the distributed element, we wrote down the voltage and current relations, taking the limit as the size of the circuit tends to zero so it is valid for any arbitrary frequency. This now complete the first part of analysis of transmission line. So we shall deal with more application of transmission line, we will go and discuss now a graphical representation of transmission line or graphical tool for analyzing the problems of transmission line, and later on go to application of transmission line at high frequencies. 

\begin{tabular}{l c l}
{\bf Lossless}&   	$\leftrightarrow$  & {\bf Low-loss}  \\ 
$\alpha$ = 0 &  & $\alpha = \dfrac{1}{2}R\sqrt{\frac{C}{L}} + \dfrac{1}{2}G\sqrt{\frac{C}{L}}$  \\
$\beta = \omega\sqrt{LC}$ & & $\beta = \omega\sqrt{LC}$\\
$\gamma = j\beta$ & & $\gamma = \alpha + j\beta$\\
$Z_o = \sqrt{\frac{L}{C}}$ = real &  & 	$Z_o = \sqrt{\frac{L}{C}}\left(1 - j\frac{R}{2\omega L} - j\frac{G}{2\omega C}\right)$
\end{tabular} 

\subsection*{Generally (i.e. for both cases- lossless and low-loss)}

\begin{align*}
\Gamma_L = \frac{Z_L - Z_o}{Z_L + Z_o} = \frac{\bar{Z_L} - 1}{\bar{Z_L} + 1}
\end{align*}

\begin{align*}
\Gamma_L = \frac{V^-}{V^+},
VSWR(\rho) = \frac{1 + |\Gamma_L|}{1 - |\Gamma_L|}
\end{align*}

\begin{align*}
R_{max} = Z_o \left( \frac{1 + |\Gamma_L|}{1 - |\Gamma_L|}\right) = Z_o \rho,\\
Z_{max} = \frac{|V_{max}|}{|I_{min}|}
\end{align*}

\begin{align*}
R_{min} = Z_o \left(\frac{1 - |\Gamma_L|}{1 + |\Gamma_L|}\right) = \frac{Z_o}{\rho}, Z_{min} = \frac{|V_{min}|}{|V_{max}|}
\end{align*}

For Impedance transformation, 
$$Z(l) = Z_o\left\{\frac{Z_L\cos\beta l + Z_o\sin\beta l}{Z_o\cos\beta l + Z_L\cos\beta l}\right\} $$
$$= Z_o\left\{\frac{Z_L\cosh\gamma l + Z_o\sinh\gamma l}{Z_o\cosh\gamma l + Z_L\sinh\gamma l}\right\}	$$
When the relationship is normalized, $Z_o$ is eliminated from the expression and $Z_L$ takes the form $\bar{Z_L}$.
$$P = \frac{1}{2}\frac{|V^+|}{Z_o}\lbrace 1 - |\Gamma_L|^2\rbrace --\text{Real power}	$$
$$P = \frac{1}{2}\frac{|V^+|}{Z_o}\lbrace 1 - |\Gamma_L|^2\rbrace + j2Im\lbrace \Gamma_L e^{- j2\beta l}\rbrace --\text{Complex power}$$
$$V^+ = \dfrac{Z_L^{\textquotesingle}V_s e^{-j\beta l}}{(Z_s + Z_L^{\textquotesingle})(1 + \Gamma_L e^{-j2\beta
l})}$$