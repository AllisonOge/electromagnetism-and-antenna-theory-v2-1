\chapter{Maxwell's Equations in a source free, unbound, isotropic and homogeneous medium}\label{lec:lec21}
 \textbf{OBJECTIVES}
\begin{itemize}
	\item To define unbound,isotropic and homogeneous medium.
	\item To find the generalized forms for wave equation.
	\item To be able to identify conditions of electric fields in a plane.
	
	
\end{itemize}


In the previous chapter, we discussed Maxwell's equations and the boundary conditions. Now we are going to investigate the solutions to Maxwell's equations.

Let us consider a medium that is source free, that is, there are no charges and no current and the region is unbound, and that are no boundary conditions to be applied. Now the question that comes to mind is, \emph{what form will the electrical and magnetic field take in this medium?} At this point we ask the question of how will the electric and magnetic field be formed in such a medium (we said no source). The question now will be what will be the relationship between the electric and magnetic fields in this unbound surface medium.

So we will investigate the relationship between electric and magnetic fields in this unbound source-free medium. Let us also assume the solution we get from this equation will be simple and consistent with Maxwell's equations, so we look for solutions consistent with Maxwell's equations. What we mean by this is, if we look at the simplest solution which is consistent with the constraint, if we take the simplest example that if a function was known at one point, then the simplest solution will be a function constant passing through that point. If we know the function value at two points, then it will be a linear function passing through the two-point and so on. Of course, we have two points and we have infinite functions which can pass through these points, however, we said earlier that we would accept the simplest solution which will be a linear solution consistent with the value of the function at these two points. This is the same thing we going to do for the solution to Maxwell's equations. 

We will take the simplest solution that is consistent with Maxwell's equation, if it is not, we increase the complexity of the solution until it is complex and consistent with Maxwell's equations.

Let us consider an unbound, isotropic, homogeneous region (no sources).
\begin{enumerate}[(i)]
\item \textbf{Unbound}: No boundary condition is to be applied.
\item \textbf{Isotropic}: That is the permittivity and permeability are scalar quantities.
\item \textbf{Homogeneous}: That is, the medium permittivity and permeability are not functions of space and time.(where  in such a  medium,  $\mu $ and    $\epsilon  $  are    constant) 
\end{enumerate}
Thus, charge density $\rho=0$ and current density $\vec{J}=0$

So we can write Maxwell's equation as
\begin{align}
\nabla\cdot\vec{D}=0
\end{align}

\begin{align}
\nabla\cdot\vec{B}=0
\end{align}

\begin{align}
\nabla\times\vec{E}=\frac{-\vec{\partial B}}{\partial t}
\end{align}

\begin{align}
	\nabla\times\vec{H}=\frac{\vec{\partial D}}{\partial t}J
\end{align}

where J=1(source free medium)

\begin{align}
\nabla\times\vec{H}=\frac{\vec{\partial D}}{\partial t}
\end{align}

Recall 

$\vec{B}=\mu\vec{H}$

$\vec{D}=\epsilon\vec{E}$

$\varepsilon$ and $\mu$ are scalar quantities. However, the $\mu$ and $\epsilon$ are in an unbound medium which is similar to free space. The permittivity and permeability for free space are denoted by $\mu_{0}$ and $\epsilon_{0}$. Let us consider a dielectric medium which is unbound but not necessarily free space. We have that the medium has $\mu$ and $\epsilon$ but no boundary to this medium. \emph{What will be the relationship between electric and magnetic fields in this medium?} 


if $\mu=\mu_{0}$ and $\epsilon=\epsilon_{0}$ we would get a solution corresponding to free space

Substituting $\vec{B}$ and 	$\vec{D}$ in equation 21.3 and 21.4
\begin{align}
\nabla\times\vec{E}=\frac{-\vec{\partial B}}{\partial t}=\frac{-\mu\vec{\partial H}}{\partial t}
\end{align}
\begin{align}
\nabla\times\vec{H}=\frac{\vec{\partial D}}{\partial t}=\frac{\epsilon\vec{\partial E}}{\partial t}
\end{align}
In electrical engineering, we solve the problem of periodic signals and the periodic signal can be decomposed into its Fourier series. If we find out the behaviour of the system for a sinusoidal signal, we can always find the response of the system for periodic signals. So let us assume without losing generality that we investigate the problem here for the time-harmonic fields. Let us assume that both the electric field and magnetic field are functions of time.

That is, $\vec{E}$ and $\vec{H}$ are time-varying functions and are sinusoidal functions of time.

$\vec{E} \sim e^{j\omega t}$

$\vec{H} \sim e^{j\omega t}$

where 
$\omega$ = the angular frequency.
So far our earlier assumption when we differentiate a time-varying function of $e^{j\omega t}$

$\frac{\partial}{\partial t} \equiv j\omega$, 
$\frac{\partial^2}{\partial t^2} \equiv j\omega \bullet j\omega \equiv j^2\omega^2=-\omega^2$

Therefore for time-harmonic case and a source free medium
\begin{align}
\nabla\cdot\vec{D}=\nabla\cdot \epsilon \vec{E}=\epsilon \nabla\cdot\vec{E}=0
\end{align}
\begin{align}
\nabla\cdot\vec{B}=\nabla\cdot (\mu\vec{H})=\mu \nabla\cdot\vec{H}=0
\end{align}
\begin{align}
\nabla \times \vec{E}=-j\omega\mu\vec{H}
\end{align}
\begin{align}
\nabla\times\vec{H}=j\omega\epsilon\vec{E}
\end{align}
The equation for an unbound, homogeneous medium is derived from general Maxwell's equation. Looking at the last two equations, the space derivative of the electric field is relative to the time derivative of the magnetic field, so essentially now we are looking for the solution of these four equations. If we look at $\nabla\times\vec{E}=-j\omega\mu\vec{H}$ and $\nabla\times\vec{H}=j\omega\mu\vec{E}$, $\nabla\times\vec{E}$ is dealing with the partial derivative of the electric field and is related to the time derivative $-j\omega\mu\vec{H}$ of the magnetic field. If we recall this equation is similar to what we had for (Transmission Lines). For the transmission line equation replacing electric field by voltage and magnetic field by current, then for transmission lines, we derived the voltage equal to $-j\omega LI$. So $\mu$ which is permeability has unit H/m, electric field E has unit V/m, and H is A/m. So if we take the per meter out the E will be volts,$\mu$ will be H and it will be A. So $\nabla\times\vec{E}=-j\omega\mu\vec{H}$ is identical to the equation we got for transmission lines. Exactly similar thing happens in $\nabla\times\vec{H}=j\omega\mu\vec{E}$. That is the space derivation of electric current is related to $j\omega E$ which is like capacitance and current. $\nabla\times\vec{E}=-j\omega\mu\vec{H}$ and $\nabla\times\vec{H}=j\omega\mu\vec{E}$, are generalized forms of the equations we got for transmission lines. So here we are having a three-dimensional space. Also here we are having electric and magnetic fields which are not scalar as we had in transmission lines.

So we have these vector quantities $\nabla\times\vec{E}=-j\omega\mu\vec{H}$, which have a derivative in 3D space denoted by the $\nabla$ operator. So transmission lines case which we have discussed in a special case of this generalised 3D case. Also as we have found out when analyzing transmission lines. These solutions are coupled equations, so that the electric field $\vec{E}$ is related to the magnetic field and the magnetic field is related to the electric field.

So if we want a solution for both electric and magnetic field, we have to decouple the two equations $\nabla\times\vec{E}=-j\omega\mu\vec{H}$ and $\nabla\times\vec{H}=j\omega\mu\vec{E}$, we recall in transmission lines we have done that by the derivative of each of the equation and substituting for the known equation. We also noted that both $\vec{H}$ and $\vec{E}$ are vectors quantities, that we define the space derivative, we ask what is meant by space derivative since $\vec{H}$ and $\vec{E}$ are vectors quantities, in terms of simple space function the derivatives was $\frac{\partial}{\partial x}$, $\frac{\partial}{\partial y}$, $\frac{\partial}{\partial z}$. However we are talking about space derivatives in 3D space, we can operate space derivative $\nabla\times\vec{E}=-j\omega\mu\vec{H}$ and $\nabla\times\vec{H}=j\omega\mu\vec{E}$ which would like divergence operator or like curl operator on $\nabla\times\vec{E}$ and $\nabla\times\vec{H}$. 

So either we find the divergence of the equation or we find the curl of the equation and this is the two ways to find the space derivatives of the equation.

Divergence of $\nabla\times\vec{E}$ is identically zero. So that $\nabla\cdot(-j\omega\mu\vec{H})=0$ from $\nabla\times\vec{E}=-j\omega\mu\vec{H}$, take derivatives of both sides $\nabla\cdot(\nabla\times\vec{E})=\nabla\cdot(-j\omega\mu\vec{H})$ or $\nabla\cdot(-j\omega\mu\vec{H})=0$ similar to $\mu\nabla\cdot\vec{H}=0$. We would have the same thing happen if we carry out divergence on $\nabla\times\vec{H}=j\omega\mu\vec{E}$ on both sides, the equation reduces to $\epsilon\nabla\cdot\vec{E}=0$ which we had already.

So divergence as a space derivative carried out on the coupled equation $\nabla\times\vec{E}=-j\omega\mu\vec{H}$ and $\nabla\times\vec{H}=j\omega\mu\vec{E}$ did not give us any other different expression we have not had before. So we try to take the curl of both sides of the equation which is another form of space derivatives.

It will be observed that the coupled equation for voltage and current in the transmission is analogous to the space equation for the electromagnetic wave. Essentially, equations (21.7) and (21.8) are generalized from these equations.

No new equation will be formed when taking the divergence, so let us take the curl of equation(21.9)
\begin{align}
\nabla\times\nabla\times\vec{E}=-j\omega\mu\nabla\times\vec{H}
\end{align}
but 
\begin{align}
\nabla\times\nabla\times\vec{E}=\nabla(\nabla\cdot\vec{E}) - \nabla^2\vec{E}
\end{align}
\begin{align}
-j\omega\mu\nabla\times \vec{H}=-j\omega\mu(j\omega\epsilon\vec{E})
\end{align}
from equation (21.7) $\nabla\cdot\vec{E}=0$ (for homogeneous)
\begin{align}
\nabla^2\vec{E}=j^2\omega^2\mu\epsilon\vec{E}=-\omega^2\mu\epsilon\vec{E}
\end{align}
Let us take a look at equation(21.10)

$\nabla\times\vec{H}=j\omega\epsilon\vec{E}$

Take curl of both side
\begin{align}
\nabla\times\nabla\times\vec{H}=j\omega\epsilon\nabla\times\vec{E}
\end{align}
\begin{align}
\nabla\times\nabla\times\vec{H}=\nabla(\nabla\cdot\vec{H})-\nabla^2\vec{H}
\end{align}
where $\nabla\cdot\bar{H}=0$
\begin{align}
\nabla\times\nabla\times\vec{H}=-\nabla^2\vec{H}
\end{align}
\begin{align}
\nabla^2 \hat{H}=-\omega^\epsilon \hat{H}
\end{align}

Equation(1.14) governs the electric field while Equation(1.18) governs the magnetic field, and they are similar. Each of these equations consists of three equations, i.e three components of $\vec{E}$ in cartesian co-ordinate $(E_{x}, E_{y}, E_{y})$ that satisfies the equation and that of $\vec{H}$ $(H_{x},H_{y},H_{y})$ satisfies the equation.

The generalised form for wave equation gotten in the transmission line is
\begin{align}
\nabla^2\cdot\vec{E}=-\omega^2\mu\epsilon\vec{E}
\end{align}
\begin{align}
\nabla^2\vec{H}=-\omega^2\mu\epsilon\vec{H}
\end{align}

The above equation is the general case of Ohms law,while Telegraphers equation is the specialised form of it


So what we want to get is the solution to the set of equations that solve Maxwell's equations.

We want to proceed from the simplest possible solutions and ask if the solution is consistent with Maxwell's equations.

\section{Case 1: the electric field is uniform}

Let say the electric field ($\vec{E}$ ) is uniform in the three dimensional space,

Therefore $\vec{E}=\vec{K}_{0}$

Actually $\vec{E}=\vec{k_{0}}e^{j\omega t}$ we do not always put $e^{j\omega t}$ as we assume it is implicit in all the solutions that we are defining. So when we require an instantaneous value we always multiply by $e^{j\omega t}$. For $\vec{E}=\vec{K}_{0}$ to be consistent with Maxwell's equation it must satisfy the equations which we have derived. So we can substitute $\vec{E}$ into equation (21.11).

\begin{align}
\nabla^2\vec{K_{0}}=-\omega^2\mu\epsilon\vec{K_{0}}
\end{align}
\begin{align}
\nabla^2\vec{K_{0}}=0
\end{align}

$\omega^2\mu\epsilon\vec{K}=0$

$\omega\neq0$ (for a time vary field)

$\mu\epsilon=0$

$\vec{K_{0}}\equiv0$

$\vec{E}=\vec{K}=0$ Meaning no electric field at all, in other words, a uniform electric field is not consistent with Maxwell's equation. So if we are talking about a time-varying field, then a uniform electric field in three-dimensional space is not consistent with the wave equation or Maxwell's equation.

Therefore uniform three-dimensional time-varying fields can not exist.

\section{Case 2: the electric field is not uniform}
Let us say the electric field ($\vec{E}$) is uniform in a plane,not  in all three-dimensional space but uniform in a plane and let us choose a coordinate plane in such a way that the field is oriented in the x-axis. Without losing a sense of generality, we have defined which way the coordinate system is and which direction the electric and magnetic fields are oriented. So we have the freedom to choose the coordinate system such that the x-axis is oriented in the direction of the electric field.

i.e $\vec{E}={E_{0}}(x)\hat{x}$

In this case, we are assuming that the electric field $\vec{E}$ is constant in the direction of the x-axis or a plane perpendicular to the x-axis, let us consider the former. For a variation of $\vec{E}$ in the yz plane and the other in the xz plane. $\vec{E}$ is uniform in xy plane or z plane. Now for variation in the yz or xz plane, let us use the xz plane only and say $\vec{E}$ is dependent only on x for variation instead of (x and z) to simplify the problem. So we have $\vec{E}={E_{0}(x)}\hat{x}$, if this is the solution.

\begin{align}
Consider\quad\nabla\times\vec{E}=-j\omega\mu\vec{H}
\end{align}


\[
\begin{vmatrix}
$\^{x}$ & $\^{y}$ & $\^{z} $\\
$$\frac{\partial}{\partial x}$$ & $0 $ & $0 $\\
$$E_{0}(x)$$ & 0 & 0
\end{vmatrix} =
-j\omega \mu \vec{H}
\]




\[
\begin{vmatrix}
$\^{x}$ & $\^{y}$ & $\^{z} $ \\
$$\frac{\partial}{\partial x}$$ & $0 $ & $0 $ \\
$$E_{0}(x)$$ & 0 & 0
\end{vmatrix} =0 \Longrightarrow j\omega\mu\vec{H}\equiv0
\]

So since we assume that $\vec{E}$ is not varying in y and z direction,

$\frac{\partial}{\partial y}$ = 0,
$\frac{\partial}{\partial z}$ = 0

Note: $\vec{E}={E_{0}(x)\hat{x}}$ is uniform in the yz plane only but not uniform in the xy and xz planes. With $j\omega\mu\vec{H}=0$, in time varying field $\omega\neq0$, $\omega\mu\neq0$, $\Longrightarrow$ $\vec{H}=0$. That means these electric fields would exist without any magnetic field. However we know from Maxwell's equation that electric and magnetic fields are coupled, so if the magnetic does not exist in the time-varying case, then the electric field cannot also exist. Since there is no magnetic field in this case, an electric field cannot exist, so the electric field $\vec{E}$ is identically zero.

So if the electric field is uniform in a plane perpendicular to its orientation, then the field has to be identically zero because the magnetic field is zero. So we use the second conclusion that the electric field which is uniform in a plane perpendicular to the x-direction can not exist in an unbound medium. Let us consider the third possibility.

\section{Case 3: the electric field is uniform in a plane}
$\vec{E}$ is uniform in a plane containing the $\vec{E}$ vector. Let us assume this is constant in the x-y plane and have variation only in the z-direction $\vec{E}=E_{0}(z)\hat{x}$ unlike the previous case, $\vec{E}$ is a function of z so $\frac{\partial}{\partial z}\neq0$, but $\frac{\partial}{\partial x}$ and $\frac{\partial}{\partial y}=0$, so the $\nabla\times\vec{E}=-j\omega\mu\vec{H}$

\begin{dmath*}
%\begin{bmatrix}
%$\^{x}$ & $\^{y}$ & $\^{z}$\\
%0 & 0 & \pdv{}{z} \\
%$E_{0}(z)$ & 0 & 0
%\end{bmatrix}=	
\begin{vmatrix}
$\^{x}$ & $\^{y}$ & $\^{z}$\\
0 & 0 & \pdv{}{z} \\
$$E_{0}(z)$$ & 0 & 0
\end{vmatrix}=-j\omega\mu\vec{H}=\hat{x}(0) - \hat{y}\frac{E_{0}(z)}{\partial z} + \hat{z}(0)
\end{dmath*}			
$-\frac{\partial E_{0}(z)}{\partial z}\hat{y}$=$-j\omega\mu(H_{x}\hat{x} + H_{y}\hat{y} + H_{z}\hat{z} )$
So we have only the y component of the magnetic field $H_{y}$ = $\frac{1}{j\omega\mu}$ $\frac{\partial E_{0}(z)}{\partial z}$

so if we consider the electric field oriented in the x-direction and assume that it is constant in the x-y plane, then the corresponding magnetic field is oriented in the y-direction yet the magnetic field does exist but in the previous two cases, the magnetic field did not exist. We saw that the electric field did not satisfy the wave equation for the time-varying field, $\vec{E}$ was identically zero. In the second case when the electric field was constant in a plane perpendicular to its direction, it did not have a corresponding magnetic field, because the magnetic field was partially zero. For that same reason electric field for the time-varying field had to be zero.   
In this particular case however, we find that the electric field which is oriented in the x direction and varies in the z direction i.e constant in the x-y plane has a corresponding magnetic field and since E is a function of z i.e E is not constant as a function of z, the quantity $\frac{\partial {E_z}}{\partial {z}}$ is a finite quantity. We have a magnetic field $H_{y}$ = $\frac{1}{j\omega\mu}$ $\frac{\partial E_{0}(z)}{\partial z}$ associated with an x oriented electric field, so we now have three important conclusions. First, time-varying fields with uniform fields in three-dimensional space can not exist. Secondly, a field which is constant in a plane perpendicular to its direction also cannot exist if the fields are time-varying. So the simplest possible fields which can exist in this medium are the ones which are constant in a plane parallel to the direction of the vector fields.

So in general, we state that if the field is constant in any plane, perpendicular to that plane, the field has variations an electric and magnetic field will coexist in that medium. That is why for time-varying fields, this is the simplest solution. We can now proceed to get the solution of the wave equation in this particular case. So we have an electric field oriented in the x direction and it's a function of z only. $\vec{E}=E_(z)\hat{x}$ with $e^{j\omega t}$ being implicit, hence not explicitly added to the expression as we said earlier. We substitute the field in the wave equation to be $\nabla^2\vec{E}=-\omega^2\mu\epsilon\vec{E}$ since we have only the x component in $\vec{E}$, the equation should be satisfied by the x component.

\begin{align}
\nabla^2\vec{E_{x}}=-\omega^2\mu\epsilon\vec{E_{x}} =( {\frac{\partial^2}{\partial x^2} + \frac{\partial^2}{\partial y^2} + \frac{\partial^2}{\partial z^2}})\vec{E_{x}}
\end{align}

Hence E is a function of z only.
$\frac{\partial^2}{\partial x^2}$ and $\frac{\partial^2}{\partial y^2}$ = 0 and $\frac{\partial^2 E_x}{\partial z^2}$=$-\omega^2\mu\epsilon E_{x}$ but since E is not varying as a function of x and y, so $E_x$ is a function of z only, so that we replace the partial derivative with the full derivative we got.

$\frac{d^2 E_x}{d z^2}$=$-\omega^2\mu\epsilon E_{x}$ this is identical to the transmission line equations we have solved for and it is now a 1-dimensional case. So we are simply saying that $E_x$ is now a scaler quantity. So if we replace $E_x$ with V, we get the equation for the transmission line. So in this case, the simplest solution which we get in the unbound medium is identical to the solution which we get for the transmission line. Since we have investigated the transmission line in detail, most of the concepts which we have done for the transmission line will be applied to this case also. So in the transmission line when we are investigating, we define the quantity $-\omega^2\mu\epsilon$ as $(propagation\hspace{0.15cm} constant)^2$, so with the same understanding, we can call $-\omega^2\mu\epsilon$ as propagation constant squared. So following the convention and notation which we have for the transmission line, we define $\gamma^2$=$-\omega^2\mu\epsilon$ or $\gamma$=$\sqrt{-\omega^2\mu\epsilon}$ = $j\omega\sqrt{\mu\epsilon}$. If we go by the way $\gamma$ was defined ealier, $\gamma$=$\alpha$ + $j\beta$, then $\alpha$ was attenuation constant and $\beta$ = phase constant. So that $\gamma$=$\alpha$ + $j\beta$ = $j\omega\sqrt{\mu\epsilon}$, this means that for this case $\alpha$ which is attenuation constant is zero and  $\beta$=$j\omega\sqrt{\mu\epsilon}$.
We recall in the transmission line that $\alpha=0$, this represents a lossless transmission line, what this physically means was that when the wave travels on the transmission line, the amplitude of the wave does not decrease, it has a constant amplitude, this is the same thing that is represented here. In this case, the solution of the wave equation is identical to the transmission line solution with $E_x$ for V and the amplitude of the wave should not vary as a function of the distance i.e. in the direction z. So this represents now in terms of transmission line terminology that we developed, a lossless propagation on the transmission line and the solution which we have gotten from the transmission line is a second-order differential equation after defining the parameter is
\begin{align}
\frac{d^2E_x}{dz^2}+\beta^2E_x=0
\end{align}
The solution to this is easily written as

$E_x(z)=E^+_xe^{-j\beta z}+E^-_xe^{j\beta z}$ This is similar to the transmission line for the simplest case of wave propagation in an unbound medium.
We get an electric field which is a combination of $E_x^+$ and $E_x^-$ and we know from the transmission line earlier that $E_x^+$ $e^{-j\beta z}$ represents a forward travelling wave in the z-direction and $E_x^-$ $e^{-j\beta z}$ represents a backward travelling wave in the z-direction. So we get a solution for the electric field which is a combination of two travelling waves of amplitude $E_x^+$ and $E_x^-$ travelling in opposite directions in positive z-direction and negative z-direction.

ADDENDUM: $E_0(x)\hat{x}$ and $E_0(z)\hat{x}$, why it exists for one and not the other using vector field plot to find where you get rotation $\nabla\times\vec{E}$.
