\chapter{Reflection from a Conducting Boundary}\label{lec:lec34}
In this chapter, we will discuss the reflection of a uniform plane wave from a conducting boundary. This will lay a foundation for a structure called a \emph{Waveguide}. A waveguide is basically a linear structure that conveys or guides electromagnetic waves between its endpoints.
\begin{figure}[h]
\centering
\includegraphics[width=1\linewidth]{\pathtopartone/graphics/"wave guides"}
\caption{A waveguide}
\end{figure}

The phenomenon of reflection from a conducting surface is considered in terms of the solutions of Maxwell equations where amplitudes of reflected and transmitted waves are found as functions of an incident wave and the conductivity of the plane. This conducting plane lies between two distinct media: 
\begin{enumerate}[(i)]
\item An ideal dielectric medium.
\item An ideal conductive medium.
\end{enumerate}
\begin{figure}[h]
\centering
\includegraphics[width=1\linewidth]{\pathtopartone/graphics/dielectric_conductor_boundary}
\caption{Conducting plane between two media}
\label{fig:plane}
\end{figure}

With reference to figure~\ref{fig:plane}, we have a horizontal conductive plane which divides the medium into two parts. Below the plane is an ideal conductor with an infinite conductivity ($\sigma _2 =\infty$) and above the plane is an ideal dielectric with conductivity $\sigma_1 = 0$, permeability $\mu _1$ and	permittivity $ \varepsilon _1$. This plane or boundary can therefore be referred to as a \textbf{Dielectric conductor boundary}.

There can be no wave propagation in the conductive medium as time-varying fields in a conductor are zero due to its infinite conductivity. So the wave can only be propagated from the dielectric medium and then incident on the conductive plane boundary.

As shown in figure~\ref{fig:plane}, the wave $\boldsymbol{E_i}$  is incident from the dielectric side making an angle $ \theta $ with the plane but it is not propagated to the conductor side because of the infinite conductivity of the conductor, hence no wave is transmitted. The wave $E_r$ is however reflected at an angle $ \theta $ with the normal of the plane\footnote{
From the law of reflection, the angle of incidence is equal to the angle of reflection.
}.

Again we can consider the case of perpendicular and parallel polarizations and carry out the analysis of the boundary as we have done for a dielectric media interface. However, this is a simpler case as we only have two waves(incident and reflected waves) to match the boundary conditions.

\section{Perpendicular Polarization}\index{Perpendicular Polarization}
Taking the perpendicular polarization, $E_i$ and $E_r$ are oriented in the y direction and by using a Poynting vector, we get the direction for the magnetic fields $H_i$ (incident magnetic field) and $H_r$ (reflected magnetic field). We can resolve the magnetic field into two components which are either \textbf{parallel} or \textbf{perpendicular} to the plane:

\begin{figure}[h]
\centering
\includegraphics[width=1.02\linewidth]{\pathtopartone/graphics/perpendicular_polarization_at_dielectric_conductor_interface}
\caption{Magnetic Field}
\label{fig:fields}
\end{figure}

\begin{enumerate}[(i)]
\item $ H_isin\theta $ and $ H_rsin\theta $  are the normal (perpendicular) components to the plane.
\item $ H_icos\theta $ and $ H_rcos\theta $ are the tangential (parallel) components of the plane.
\end{enumerate}
Since we are having a conducting boundary, we will have the presence of surface current, hence, we can either use the boundary conditions with appropriate surface current and the tangential component of the magnetic fields or use only the boundary condition for the perpendicular components of the magnetic fields without worrying about the surface current. Since there is no wave propagation in the conducting medium, the time-varying fields $E_t$ and $H_t$ are zero, and the boundary conditions now have to be satisfied by only the incident and reflected waves.

Writing the expressions for the incident and reflected Electric and Magnetic fields and taking the appropriate components of these fields to satisfy the boundary conditions, we have that the ratio of the electric to the magnetic field for both the incident and reflected waves is equal to the intrinsic impedance $\eta $. That is $ \eta $ is the intrinsic impedance given as $\eta $ = $\sqrt{\frac{\mu}{\varepsilon}}$ = $\dfrac{E_r}{H_r} = \dfrac{E_i}{H_i}$.

Note that we are not using subscript 1 for the dielectric medium(medium 1) or subscript 2 for medium 2 as there is no wave propagation in the conducting medium(medium 2).

\subsection{Electric Field}
\begin{align}
\bar{E}_i = E_i e^{-j\beta(-x\cos\theta + z\sin\theta)} \hat{y}
\label{eqn:elecincid}
\end{align}
\begin{align}
\bar{E}_r =E_r  e^{-j\beta ( x\cos\theta + z\sin\theta)} \hat{y}\footnotemark
\label{eqn:elecref}
\end{align}
\footnotetext{
$\hat{y}$ signifies the field is in the y direction
}
Where $\bar{E}_i$ is the incident electric field and $\bar{E}_r$ is the reflected electric field. $E_i$ and $E_r$ are the amplitudes of the electric fields, $e^{-j\beta}$ is the phase function and $\beta$ is the propagation constant given as $\omega \sqrt{\mu\varepsilon}$ 

\subsection{Magnetic Field}
\begin{align}
\bar{H}_i = (-H_i sin\theta \hat{x} - H_i cos\theta \hat{z}) e ^{-j\beta( -xcos\theta + zsin\theta)}
\label{eqn:magincid}
\end{align}
\begin{align}
\bar{H}_r = (-H_r sin\theta \hat{x} + H_r cos\theta \hat{z}) e ^{-j\beta( xcos\theta + zsin\theta)}
\label{eqn:magref}
\end{align}\footnote{$\hat{x}$ signifies it's in the x direction and $\hat{z}$ signifies it's in the z direction}
where $\bar{H}_i$ is the incident magnetic field and $\bar{H}_r$ is the reflected magnetic field. $H_i$ and $H_r$ are the amplitudes of the magnetic fields. As stated earlier $\frac{E_r}{H_r} = \eta = \frac{E_i}{H_i}$, hence, 
\begin{align}
H_i = \frac{E_i}{\eta}
\label{eqn:wavemagincid}
\end{align}
\begin{align}
H_r = \frac{E_r}{\eta}
\label{eqn:wavemagref}
\end{align}
From equations~\eqref{eqn:wavemagincid} and~\eqref{eqn:wavemagref}, we can rewrite the expressions for $H_i$ and $H_r$ in equations~\eqref{eqn:magincid} and~\eqref{eqn:magref} in terms of $E_i$ and $E_r$ respectively;
\begin{align}
\bar{H}_i = - \frac{E_i}{\eta} (sin\theta \hat{x} + cos\theta \hat{z})e^{-j\beta (-xcos\theta + zsin\theta)}
\end{align}
\begin{align}
\bar{H}_r =  \frac{E_r}{\eta} (-sin\theta \hat{x} + cos\theta \hat{z})e^{-j\beta (xcos\theta + zsin\theta)}
\end{align}
So now that we have the expressions for the electric and magnetic fields of the incident and the reflected waves, we need to obtain the boundary conditions which are appropriate for the plane.
The boundary conditions are:
\begin{enumerate}[(i)]
\item The tangential component of the electric field should be continuous across the plane i.e. $\bar{E}_{tangential}$ must be continuous.
\item The normal component of the magnetic field should also be continuous across the plane i.e. $\bar{H}_{normal}$ must be continuous.
\end{enumerate} 
In figure~\ref{fig:fields}, we can see that the tangential component to the plane is the electric field. At this point, if we equate $x = 0$ in equations \ref{eqn:elecincid} and \ref{eqn:elecref}, the sum of the two electric fields should be zero because the fields are continuous and there are no fields in the conductive medium.
\begin{align}
\bar{E} = (E_i + E_r) e^{-j\beta(zsin\theta) }\hat{y} = 0
\end{align}
Such that it simplifies to $E_i = - E_r$ i.e, $\frac{E_r}{E_i}$ = -1 \textbf{(reflection coefficient)}. 

So in this case, the electric field reflection coefficient is always equal to -1. Similarly, the sum of the two normal components of the
magnetic fields at x equal to 0 should be 0 because the normal component is also continuous at the boundary.

Recall that, in transmission line analysis, the reflection coefficient is -1 for a \textbf{short-circuited load}. This means that the conducting boundary is identical to the short circuit condition on a transmission line. Therefore, the dielectric medium on which the wave is propagated is analogous to a transmission line. When the electric wave reaches this ideal conductive boundary, it is completely reflected from the boundary with a phase reversal (phase difference of 180 degrees). So this boundary essentially behaves like a short circuit in the transmission line terminology.

Now, we want to know the kind of field patterns that are created when the electric field is completely reflected in the dielectric medium.

\section{Electric fields in the dielectric medium}
Recall that $E_{i}$ = -$E_{r}$

By superimposition of the two waves,

$\bar{E}= \bar{E}_i + \bar{E}_r$

Substitute $E_i$ = - $E_r$ and add equations~\eqref{eqn:elecincid} and~\eqref{eqn:elecref}
\begin{align}
\bar{E}= \bar{E}_i e^{-j\beta z\sin\theta} (e^{j\beta x\cos\theta }- e^{-j\beta x\cos\theta}) \hat{y}
\label{eqn:totalelectricfield}
\end{align}
Recall that $e^{jx} - e^{-jx} = 2jsinx$\\ 
From that analogy,\\ 
$e^{j\beta xcos\theta} - e^{-j\beta xcos\theta} = 2jsin(\beta xcos\theta)$

Substituting into equation~\ref{eqn:totalelectricfield}
\begin{align}
\bar{E}=2j \bar{E}_i sin(\beta xcos\theta) e^{-j\beta zsin\theta} \hat{y}
\label{eqn:totalelectricfield2}
\end{align}
From equation~\ref{eqn:totalelectricfield2}, we have the electric fields with a sinusoidal variation in the x-direction and a phase term which is in the z-direction. This means that the electric field has a \textbf{standing wave} behaviour in the x direction because the function {sin({$\beta xcos\theta$})} does not have a phase but it has an amplitude variation which is the nature of a complete standing wave. So equation~\ref{eqn:totalelectricfield2} represents something like a standing wave which is in the x direction and a travelling wave which is given by this term \textbf{$e^{-j\beta zsin\theta}$} in the z-direction. The equation thus gives us a \textbf{standing wave} in the x-direction and a \textbf{travelling wave} in the z-direction. Therefore, the wave is a composite phenomenon of the incident and the reflected wave making it a \textbf{complex wave}.

A complex wave is the combination of a standing wave which is in a direction perpendicular to the plane or boundary and a travelling wave which is in the direction of the plane.

The same can be obtained for the magnetic fields when we combine the incident and reflected magnetic fields. We therefore
would have an expression for standing and travelling magnetic waves.

\section{Magnetic fields in the dielectric medium}
Taking the two components of the magnetic fields in the x and z direction
$\boldsymbol{H_x}$ and $\boldsymbol{H_z}$ respectively, and adding equations \ref{eqn:magincid} and \ref{eqn:magref}, we obtain
\begin{align*}
H_x = \frac{-E_i}{\eta} (sin\theta \hat{x} + cos\theta \hat{z}) e^{-j\beta( -xcos\theta + zsin\theta)} +\\ 
\frac{E_r}{\eta} \left(-sin\theta \hat{x} + cos\theta \hat{z}\right) e^{-j\beta( xcos\theta + zsin\theta)}
\end{align*}
\begin{align}
H_x = \frac{-E_i}{\eta}(sin\theta e^{j\beta( xcos\theta)} ) - \frac{E_r}{\eta}\sin\theta e^{-j\beta( xcos\theta)} (e^{-j\beta zsin\theta})
\label{eqn:totalmagfieldx}
\end{align}
Recall that $E_i$ = -$E_r$ and substituting into equation~\ref{eqn:totalmagfieldx}
\begin{align}
Hx = \frac{-E_i}{\eta}(sin\theta)( e^{j\beta( x\cos\theta)} - e^{-j\beta x\cos\theta}) (e^{-j\beta zsin\theta})
\label{eqn:totalmagfieldx2}
\end{align}
Recall that $e^{jx} - e^{-jx} = 2jsinx$. From that analogy, $e^{j\beta xcos\theta} - e^{-j\beta xcos\theta} = 2jsin\beta xcos\theta$ and substitute into
equation~\eqref{eqn:totalmagfieldx2}
\begin{align}
H_x =  -2j \frac{E_i}{\eta}sin\theta sin(\beta xcos\theta)(e^{-j\beta zsin\theta})
\end{align} 
The x component $H_x$ has similar behaviour with the electric field. That is, it has a standing wave component $(sin\beta xcos\theta)$ which is in the x-direction and a travelling wave component $(e^{j\beta zsin\theta})$ in the z-direction.

The same can be done for the z component of the magnetic field $H_z$.
\begin{align}
H_z = \frac{-E_i}{\eta}cos\theta(e^{j\beta xcos\theta} + e^{-j\beta xcos\theta}) (e^{-j\beta zsin\theta})
\label{eqn:totalmagfieldz}
\end{align}  
Recall that $e^{jx} + e^{-jx} = 2cosx$\\
From that analogy, $e^{j\beta xcos \theta} - e^{-j\beta xcos\theta} = 2cos(\beta xcos\theta)$ and substitute into equation~\eqref{eqn:totalmagfieldz}
\begin{align}
H_z = \frac{-2E_i}{\eta}cos\theta cos(\beta xcos\theta) (e^{-j \beta zsin\theta})
\label{eqn:totalmagfieldz2}
\end{align}
From equation~\eqref{eqn:totalmagfieldz2}, we see that the z component of the magnetic field Hz has a standing wave component $(cos\beta xcos\theta)$ and a travelling wave component $(e^{-j\beta zsin\theta})$. So in general, the fields in the dielectric medium are a combination of travelling and standing waves.

All of these fields travel in the positive z direction (they travel along the plane). However, note that the standing waves are in a direction perpendicular to the interface.

Now we can make some observations from the expressions which we got for the electric and magnetic fields. First, if we plot the amplitude of the electric field as a function of x when x is equal to 0 then the amplitude is zero, therefore, the electric field is zero. The same happens for the x component of the magnetic field $H_x$, when x is equal to zero the magnetic field will be 0. So whenever the electric field is zero, the x component of the magnetic field $H_x$ is also zero. It can therefore be deduced that the amplitude behaviour of $H_x$ and electric field $E_y$ is identical as a function of x. And the magnetic field component of $H_z$ is a cos function. That means it is shifted by a quarter cycle in the x direction. So whenever $H_x$ is zero, $H_z$ will be maximum and whenever $H_x$ is maximum, $H_z$ will be zero. The plot of the amplitude of the electric and magnetic fields is shown in figure~\ref{fig:amplitude}.
\begin{figure}[h]
\centering
\includegraphics[width=1\linewidth]{\pathtopartone/graphics/amplitude_of_electric_and_magnetic_fields.png}
\caption{Amplitude of electric and magnetic fields}
\label{fig:amplitude}
\end{figure}

Referring to the figure~\ref{fig:amplitude}, we have a boundary consisting of an electric field $E_y$, a magnetic field $H_x$ for the x component and a magnetic field on the z component $H_z$. We can see that both fields$ E_y$ and $H_x$ are exactly identical in behaviour whereas the z component of the magnetic field $H_z$ is shifted by a quarter cycle. So wherever $H_x$ is maximum, $H_z$ is 0 and vice versa. Since $H_z$ which is the tangential component (along the plane) is not zero, there will therefore be surface currents on the plane and the magnitude of the surface current will be equal to the tangential component of the magnetic field. When the wave is incident on the conducting boundary, the surface currents are going to get induced on the surface which is due to the tangential component of the magnetic field. Also, the normal component of the magnetic field and the tangential component of the electric field will be zero.

Now referring back to the expression for the electric field in equation~\ref{eqn:totalelectricfield2}, the electric field is zero when x is equal to zero and it will also be zero whenever the quantity $(\beta xcos\theta)$ is a multiple of $\pi$. This means that $E_y$ will be 0 when $(\beta xcos\theta)$ is a multiple of $\pi$. i.e $E_y$ will be zero when:\\
$(\beta xcos\theta) = m\pi$     where m is an integer (0,1,2, 3 ...)\\ 
Hence,\\
$ x = \frac{m\pi}{\beta cos\theta}$   \\ 
but $\beta = \frac{2\pi}{\lambda}$ \\
where $\lambda$ is the wavelength in the dielectric medium.
\begin{equation*}
x =\frac{m\pi}{\frac{2\pi}{\lambda}\cos\theta}
\end{equation*} 
Thus,
\begin{equation}
x = \frac{m\lambda}{2cos\theta}
\end{equation}
So at this distance x from the plane, the electric field will be 0. This distance is however dependent on the angle at which the wave is launched on the plane. There are multiple planes here in which the electric field is zero and since the electric field and the x component of the magnetic field $H_x$ have the same behaviour in those planes, both quantities will be zero but $H_z$ will be maximum in those planes. If we go backwards or forwards by a quarter cycle away, we will see that $H_z$ will be 0 and the quantities $E_y$ and $H_x$ will then be maximum.

Now, we know that the wave is traveling along the z direction so there must be a power flow in that direction. This essentially gives us the pointing vector in the z-direction. Note that there is no net power flow in the x direction. Hence, given that we have a conducting boundary, there will be no power flow through the boundary. So whatever power is incident on the boundary is reflected back to the source medium. The wave which is incident in a particular direction has components of propagation in the direction normal and parallel to the interface. The wave which is incident normal to the interface is completely reflected, hence, a standing wave is created. So there is no net power flow in the x direction. This implies that, given a conducting boundary, then the boundary can be used to guide a wave of energy along it, hence, the conducting boundary has the capability of guiding electromagnetic energy. So when a wave is launched at an arbitrary angle, the net power flow is always along the surface of the boundary interface. This is the principle used in making a waveguide. 

So in a wave-guiding structure, the conducting boundaries are used so that the electromagnetic energy is guided along these boundaries. Later in this book, we will see that, in those planes where the electric field was $0$(given by location $x = \frac{m\lambda}{2cos\theta}$ away from the boundary), we can insert another conducting boundary without affecting the field distribution and thus create a structure that is bounded from both sides. Within this structure, electromagnetic energy can be trapped and hence we can have a net propagation of electromagnetic energy along the structure. This structure is called the \textbf{parallel plane waveguide}  and will be considered in the next section.
