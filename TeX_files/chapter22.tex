\chapter{The ratio of electric field to magnetic field}\label{lec:lec22}
Previously we discussed the solution of  Maxwell's equation in an unbound medium. We saw that the simplest solution which can E exist in the unbound medium, is electric field which is constant in a plane parallel to the direction of the electric field vector or has a variation perpendicular to the direction of the electric field vector. so we assume in previous cases $\hat{E}=E_(z)\hat{x}$ that the electric field was constant in the xy plane. Substituting \={E} in the wave equations, we had a one dimensional equation $\dv[2]{E_x}{z}$ = $-\omega^2\mu\epsilon E_x$ which was identical to the transmission line equations. 

Then $\frac{d^2E_x}{dz^2}$ + $\beta^2 E_x = 0$ which was a second order differential equation gave $E_x(_{2}) = E_x^{+}e^{-j\beta z}$ as its general solution. This represented two wave traveling in opposite direction +z and -z which amplitude $E_x^{+}$ and $E_x^{-}$. Now we can substitute the solution of the electric field into any of the maxwell's equation to find out the relationship between electric field and magnetic field. So we can take the curl equation and substitute the electric field. The electric field is a function of z only, so that

$\frac{d}{dx}$ + $\frac{d}{dy}$ = 0.

$\hat{y} \frac{d}{dx} E_x^{+} e^{-j \beta z} + E_x^{-} e^{-j\beta z} = -jw \mu H_x^{x} + H_y^
{y}$

$H_x = H_z = 0$ and so we get only the \^{y} component.

$H_y = -j\beta E_x^{+}\frac{e^{-jz\beta}}{-jw\mu}$ + $-j \beta E_x^{-}\frac{e^{-jz\beta}}{-jw\mu}$.

So the magnitude field also have a wave variation in z having two traveling waves. 

$H_y = \frac{\beta}{w\mu}E_x^{+}jz\beta \div \frac{\beta}{w\mu}E_x^{+}jz\beta$.

For forward wave, $\frac{E_x^+}{H_y}$ = $\frac{w\mu}{\beta}$, for balanced wave 

$\frac{E_x^-}{H_y}$ = $\frac{w\mu}{\beta}$, we substitute for $\beta$ = w$\sqrt{\mu\epsilon}$
\begin{dmath*}
\frac{E_x^+}{H_y} = \frac{w\mu}{\beta} = \frac{w\mu}{\sqrt{\mu\epsilon}} = {\sqrt{\frac{\mu}{\epsilon}}}
\end{dmath*}
\begin{dmath*}
\frac{E_x^-}{H_y} = \frac{-w\mu}{\beta} = \frac{-w\mu}{\sqrt{\mu\epsilon}} = {-\sqrt{\frac{\mu}{\epsilon}}}
\end{dmath*}
${-\sqrt{\frac{\mu}{\epsilon}}}$, remember the units of $\frac{\mu}{\epsilon}$ = $\frac{H/m}{F/m}$ or $E_x^+$ has V/m and $H_y$ has A/m so that $\frac{\mu}{\epsilon}$ = $\frac{E_x^+}{H_y}$ has unit of impedance $\frac{V/m}{A/m}$  = $\frac{V}{A}$. So what ever the quantity ${-\sqrt{\frac{\mu}{\epsilon}}}$, is H is related to medium parameters of $\mu$ and $\epsilon$  and also has the unit of impedance. So this quantity is similar to what we have got for the transmission line which was characteristic impedance. Then we had the ratio of voltage and current for traveling waves which were characterized by the medium parameters.For wave propagation in 3d space, this property is referred to as INTRINSIC IMPEDANCE of the media denoted by eta $\eta$. $\eta$ = ${-\sqrt{\frac{\mu}{\epsilon}}}$. This quantity plays same role in the 3d propagation as the characteristics impedance of a transmission line (which is in 1 dimension). So we have a characteristic impedance which is now decided by the medium parameters. If we take  a specific case of the medium being free space, $\mu$ = $\mu_{o}$ and $\epsilon$ = $\epsilon_{o}$. So we substitute $\eta$ =$\sqrt{\frac{4\pi \times 10^-7}{\frac{1}{36\pi} \times 10^{-9}}}$ $\eta = 120\pi$ = $\eta_{o}$ for free space $\eta_{o} = 120\pi$$\eta$.

When we discuss transmission line, we saw the ratio of the voltage and current for a traveling wave is always equal to the characteristic impedance same statement we can make for a traveling wave in 3d space, that for this wave the ratio of electric and magnetic field $\frac{E_x^{+}}{H_y}$ or $\frac{E_x^{-}}{H_y}$ is equal to the intrinsic impedance of the medium and for free space, this intrinsic impedance is 120$\pi$ for free space. If we look at a vacuum i.e free space, the wave seems to see an impedance of $120\pi\Omega$. So a medium like vacuum or free space appears like impedance when the wave travels in the medium. So free space appears like impedance when the wave travels in this medium. So free space appears like resistance, so if the wave travels in it, the wave is essentially delivering power to this medium. So from the persecutive of the system generating thus wave, it is as if the system is delivering power to space. So space appears like resistance that consumes power from the system generating the waves point of views. So it appears as if the power is given to a resistance with a value of $120\pi\Omega$. This is interesting because space is a medium which has nothing in it, yet it appears like a resistance of a value of 377ohm. So when we get to discussions on Antennas, we see that space will be treated like impedance of 377ohms. to which power is being delivered. So once we get the intrinsic impedance of the medium, the ratio of electric and magnetic field is equal to the intrinsic impedance of the medium. That is  true for $\frac{E_x^+}{H_y}$ = $\sqrt{\frac{\mu}{\epsilon}}$, but for $\frac{E_x^-}{H_y}$ = - $\sqrt{\frac{\mu}{\epsilon}}$, we recall that we had similar situation for transmission line also that a forward traveling wave will see an impedance equal to characteristics impedance. We understand that it was as if the power was being supplied backward towards the generator. So the negative sign was the reversal of the power flow. However in this case when we are talking about E_x and H_y, we have a negative sign in $\frac{E_x^-}{H_y}$ = -$\sqrt{\frac{\mu}{\epsilon}}$ will also represent the flow of power backward. So for electric and magnetic filed which are perpendicular to each other, E in x direction and H in y direction. If H is forward traveling wave the ratio of E_x$^+$ to H_y is equal to the intrinsic impedance and if the wave is traveling backwards, the ratio is equal to the negative of the intrinsic impedance. For this x oriented electric field, all argument here will be valid in a y oriented field as far as it is constant on a plane parallel  to the direction of this electric field. So in general any electric field oriented in any direction in the xy plane can always be decomposed to the x and y component and then substitute into the equation, and we can find out the relationship between electric and magnetic fields of the components. So as we did for the analysis for the x oriented field, if we do the analysis for the y oriented field, we again get E_xpressions constant with that for x oriented fields.

\={E} = $E_y(z)$\^{y} we get $\frac{d^2E_y}{dz^2}$ + $\beta^2dE_z$ = 0 and  

$E_y(z) = E_z^+e^{-jz\beta}$ + $E_y^-e^{jz\beta}$

At this point when we find out how $E_y$ relates to H, with the curl $E_x = 0, E_z = 0$ and $\pdv{}{x}$,$\pdv{}{y} = 0$ varies with z only.

\[
\begin{bmatrix}
$\^{x}$ & $\^{y}$ & $\^{z}$\\
\frac{\partial}{\partial x} & \frac{\partial}{\partial y} & \frac{\partial}{\partial z}\\
0 & E_y(z) & 0
\end{bmatrix} =
\begin{bmatrix}
$\^{x}$ & $\^{y}$ & $\^{z}$\\
0 & 0 & \frac{\partial}{\partial z}\\
0 & E_y(z) & 0
\end{bmatrix}
\]

Now the magnetic field is not y oriented as y component is zero.
\begin{dmath*}
-\pdv{E_y(z)}{z}\hat{x} = -j\omega\mu H = -jw\mu H_x\hat{x} + H_y\hat{y} + H_z\hat{z} = H_x = H_z = 0
\end{dmath*}
\begin{dmath*}
H_x = \frac{1}{-j\omega\mu}\{\frac{-\partial y(z)}{\partial z}\} = \frac{1}{jw\mu}\{\frac{\delta y(z)}{\partial z}\}
\end{dmath*} 
$E_y(z) = E_y^+ e^{-j\beta z} + E_y^- e^{+j\beta z}$
\begin{dmath*}
H_x= \frac{1}{jw\mu} \{-j\beta E_z^+e^{-j\beta z} + -j\beta E_z^-e^{+j\beta z} \}
\end{dmath*}
\begin{dmath*}
H_x = - \frac{\beta}{w\mu} E_y^+e^{-j\beta z} +\frac{\beta}{w\mu}E_y^-e^{+j\beta z}
\end{dmath*}
Again we get the two traveling waves for the magnetic field also. Taking the ratio as we have done before 
\begin{dmath*}
\frac{E_y^+}{H_x} = -\frac{w\mu}{\beta} = -\frac{w\mu}{w\sqrt{\mu \epsilon}} = -\sqrt{\frac{\mu}{\epsilon}} = \eta,
\end{dmath*}
For backward wave
\begin{dmath*}
\frac{E_y^-}{H_x} = -\frac{w\mu}{\beta} = -\frac{w\mu}{w\sqrt{\mu \epsilon}} = -\sqrt{\frac{\mu}{\epsilon}} = \eta
\end{dmath*}
The opposite to the forward and backward wave we have taken earlier in terms of sign for the intrinsic impedance. So when we change orientation of electric field, the sign of $\eta$ changes from positive to negative for the forward wave and negative to position for the backward wave. What this essentially means is that $\frac{E_y^+}{H_x}$ or $\frac{E_y^-}{H_x}$ not only depends on the direction in which the wave is traveling, but also on the orientation of the electric and magnetic fields. So we have now E and H and the direction of propagation as a sequence, if we follow the right hand system if we have E to H in the right sequence i.e $E_x$ and $H_y$ (x to y), $\frac{E_x^+}{H_y}$ = $\eta$(positive), $\frac{E_x^-}{H_y}$ = $\eta(negative)$ but E_y and H_x (y to x), $\frac{E_x^-}{H_y}$ = $\eta(negative)$, $\frac{E_x^+}{H_y}$ = $\eta$(positive) since we are going from y to x which is opposite the right hand sequence.

so E H direction $\longmapsto E_x$ to $H_y$ gives positives

x  y z $\longmapsto E_y$ to $H_x$ gives negatives. 

So electric and magnetic field here are perpendicular to each other and have direction of propagation in the z direction. That is $E_{1}$, H and direction of propagation are all perpendicular to each other. This is electric and magnetic field are perpendicular to direction of propagation so $E_{1}$, H and direction of propagation forms an orthogonal axis. If we go from E to H in right hand sense, then we get direction of wave propagation. That E to H, then the direction of the thumb gives the direction of wave propagation. so if the largest point from wave direction to electric field, the thumb must point in the magnetic field direction. So if we follow the sequence E, H wave direction as x:y:z, For $\frac{E_x^+}{H_y^+}$ this is going from y to x and hence $\eta$ is negative sign.

For $\frac{E_y^-}{H_x}$, E_y to these gives negative sign, backward wave is another sign, so that we have multiplication of two negative sign to get positive intrinsic impedance. $\frac{E_x^+}{H_y}$ is x to y (positive) forward wave another positive i.e (+)(+)= +$\eta$. For $\frac{E_x^-}{H_y}$ we have x to y (+), backward wave negative (-), so (+)(-)=-$\eta$.

So while dealing with these vector fields in 3d space, we consider two things, the orientation of the fields and the sequence and the direction of the wave propagation. The two together will decide the ratio of electric and magnetic field as being positive $\eta$ or negative $\eta$.

However the total electric field we are going to see is a resultant of $E_x$ and $E_y$ and $H_x$ and $H_y$ for the magnetic field. So the ratio of the total electric field to total magnetic field is always positive values of $\eta$ because
\begin{center}
$E = \sqrt{E_x^2 + E_y^2}$ and $H = \sqrt{H_x^2 + H_y^2}$
\end{center}
\begin{dmath*}
\frac{|E|}{|H|}=\frac{\sqrt{E_x^2 + E_y^2}}{\sqrt{H_x^2 + H_y^2}} = \frac{\sqrt{\eta ^2H_x^2 + \eta ^2H_y^2}}{\sqrt{H_x^2 + H_y^2}}=\eta
\end{dmath*}

This means that the ratio of magnitude of electric field is always equal to the intrinsic impedance of the media. So knowing |E| we can find |H| as |H| is not an independent quantity. $\mod H$ = $\eta\mod H$. Also if we know the direction of wave propagation the $E_{1}$ H and direction of the wave propagation are perpendicular to each other.

The sequence is from E to H to wave propagation direction. That means if we know the direction of wave propagation and the direction of electric field, then the direction of the magnetic field is uniquely defined E to H, the fingers going from E to $H_{1}$ thumb points in direction of wave propagation is know, then the magnetic field is a completely dependent quantity in the electric field separately. We can just define the electric field vertically and the magnetic field separately. We can just define the electric field vectorially and the magnetic field should be perpendicular to electric field and the direction of propagation. This is the reason that whenever we do the wave analysis, we talk only about the electric fields, we know that if the electric field is described, the magnetic fields is automatically described as the magnetic field is completely dependent on the electric field. We do not have to carry separate information for the magnetic fields. Now onwards when we talk about wave propagation in this medium, we describe only the characteristic for the elastic feed and we can get the magnetic field from a knowledge of electric field.

So this phenomenon we have discussed up till now is given you on electric and magnetic field which are perpendicular and they are and they are constant in a plane perpendicular to the direction of wave propagation. This phenomenon represent was is called a UNIFORM PLANE WAVE. This wave is a TRANSVERSE ELECTROMAGNETIC UNIFORM PLANE WAVE.

This is the simplest solution which we got for fields in unbound medium. This was what was investigated before light propagation was understood, and that is what maxwell showed by analyzing that he got the velocity off light correctly by solving this problem and showing that light is a transverse electromagnetic wave. Since for most of the bulk media, the size of the medium is much longer compared to the wavelength of light , unless you go to structures where the size becomes comparable to the wavelength, we can treat the propagation of light which is an electromagnetic wave. So for those application, like wavelength, the light is treated like transverse electromagnetic wave. 

But as we see later, if the size of the structure becomes comparable to the wavelength, then the transverse nature of electromagnetic wave is lost and then we have more complE_x phenomenon which we investigate later. So what we have discussed up till now is essentially called PLANE WAVE PROPAGATION OR UNIFORM PLANE WAVE PROPAGATION.

Now if we look  at electric and magnetic field for plane wave propagation, The analysis shows that the electric and magnetic field are perpendicular to each other and they are perpendicular  to the direction of propagation E and H field cannot  change direction abruptly, they change direction with with time so as to keep E and H perpendicular. So that the new E and H still satisfies the wave equations and maxwell's equation. The solution which we have got states that E and H must be perpendicular to each other at every instant of time and must be perpendicular to the direction of wave propagation. But the solution did not say what should be the behavior of electric and magnetic field as a function of time. In fact E might change its amplitude as a function of time and the corresponding magnetic field will change its amplitude in exact as a function of time and still satisfy the solution that we have investigated. So in general we can say a uniform plane wave may have electric and magnetic field variation. That is they are uniform in a plane, but the value of electric and magnetic field at different times may be different. Since we are talking about time harmonic fields, we can assume that the electric field will be varying sinusoidally i.e some kind of periodic behavior in electric and magnetic fields. So we have the possibility of E made of $E_x$ and $E_y$ might vary as a function of time and corresponding H made if $H_x$ and $H_y$ will vary as a function of time also so we assume that E and H both vary sinusoidally at every instant of time, at every point in space, E and H remains perpendicular to each other and the ratio of electric and magnetic fields is the intrinsic impedance of the medium. Since we are saying that electric field can vary as a function of time and with location in space.

If we treat this electric field like a vector i.e with an arrow to show its direction, so with time the top of the arrow will be tracing a curve in the plane perpendicular to the wave propagation

This parameter will be captured by a parameter called the POLARIZATION of  wave. So we define the polarization of the wave as  direction of  electric field or if we treat the electric field like an arrow, the shape which the top of the electric field vector draws in a plane perpendicular to the direction of wave propagation as a function of time, is called the polarization of the electromagnetic wave. Now we investigate what is called polarization of uniform plane wave which is a variation of the direction electric field and its magnitude as a function of time.

\section{WAVE POLARIZATION}
This is a very important phenomenon in the sense that any wave you take will have a direction and magnitude of electric field which in general will vary as a function of time. So every wave will have what is called a state of polarization. Two things might happen when a wave is traveling i.e at any point in space the electric field vector might change its direction, this we call polarization. Another possibility is that as the wave propagates, at a particular location, the electric field is the same, but as it moves the electric field might change direction. This phenomenon is not polarization, this phenomenon as we see later is called Faraday's Rotation. So polarization is the orientation of electric field as a function of time at a given point in space. From point to point things might change.  At a given point in space we can say that is the polarization of the wave or state of polarization of the wave. So any wave which is generated by a signal will always have a definite polarization, so let  us look at this important concept of wave polarization again, we  see, since the electric field which propagates in the direction and varies in the xy plane can be oriented in any direction at any instant of time. So we can resolve the electric field into two components. O we can generate any arbitrary electric field by combination of two orthogonal fields one in x and the other in y directions. So we can  have $E_x$ and $E_y$ shown below. 

Combining $E_x$ and $E_y$ we can get any arbitrary field in the xy plane were the wave is propagating in the z direction. So we can combine two sinusoidally varying field $E_x$ and $E_y$, So E_x=$E_{1}\cos\omega t$ with $\omega$ as frequency i.e E_x = $E_{1}\cos\omega t$ oscillates with amplitude going to max $E_{1}$ and min $E_{1}$. $\longrightarrow$ max $E_{1}$ $\longleftarrow$ min $E_{1}$. So the top of the arrow changes as a function of time. E_y  = $E_{2}\cos\omega t + \delta$ in y direction. So we have two fields which are oriented in x and y direction, They have different amplitudes $E_{1}$ and $E_{2}$ and then have a time phase difference $\delta$, if $\delta$ is positive $E_y$ leads $E_x$ and if $\delta$ is negative $E_y$ lags $E_x$. At any instant of time, we have the resultant of $E_x$ and $E_y$ to get $E_{o}$ so the shape which E will draw as a function of time, i.e take different times and find out $E_{1}$ this will draw out some kind of curves. That curves represent the state of polarization of that wave. That is generated by combination of the two perpendicularly polarized waves $E_x$ and $E_y$. So if we launch $E_x$ and $E_y$ simultaneously into the medium, at any point in space you get the vector sum $E_x$ and $E_y$ which is E. If we have to find now the locus of the tip of the electric field, then we eliminate the time t in $E_x$ and $E_y$ and then we get the locus of the tip of the electric field vector. That gives the equation of the curve which this particular electric field vector would draw.

$E_x = E_{1}\cos\omega t$ , $E_y  = E_{2}\cos\omega t + \delta$

$\cos\omega t$ = $\frac{E_x}{E_y}$, $E_y = E_{2}\cos\omega t$ - $E_{1}\sin\omega t\sin\delta$

$\sin\omega t$ = $\sqrt{1 - \frac{E_x^2}{E_{1}^2}}$ $E_y = E_{2}\frac{E_x}{E_{1}}\cos\delta$ - $E_{2}\sqrt{1 - \frac{E_x^2}{E_{1}^2}}\sin\delta$

$E_y = \frac{E_xE_{2}\cos\delta}{E_{1}}$ - $\frac{E_{2}\sqrt{E_{1}^2 - E_x^2}}{E_{1}}\sin\delta$

$E_yE_{1}$ = $E_xE_{2}\cos\delta$ - $E_{2}\sqrt{E_{1}^2 - E_x^2}\sin\delta$

Hence we have this relationship among $E_{11}$, $E_{21}$, $E_x$ and $E_y$. That will represent the equation of the curve traced by the top of the electric field vector we shall study what are different states of polarization that have a transverse electromagnetic wave can generate and what are its implication for transmitting the signals efficiently from one point to another. So when we transmit electromagnetic waves, state if polarization plays a very important role in deciding how much power is transmitted from the system	
